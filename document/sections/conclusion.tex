\chapter{Conclusions and Future Work}
\thispagestyle{myheadings}

	In this thesis, we covered the concept of an outsourced database system and two types of adversaries --- snapshot and persistent --- that have different capabilities on an untrusted server.
	We focused on three query types --- point, range and \acrlong{knn} --- and we have gone over the many works that propose systems that execute the relevant query types in a presence of an adversary.
	For the case of range queries in a snapshot adversary model, we provided an in-depth theoretical and practical analysis, an evaluation framework, a benchmark methodology, and its application to five \acrshort{ope} / \acrshort{ore} schemes and five secure range query protocols.
	For the case of point and range queries in a persistent adversary model, we offered an efficient and secure query mechanism, \epsolute{}, along with a novel definition of security, based on \acrlong{dp}.
	For the case of \acrlong{knn} queries in a snapshot adversary model, we offered a similarity search protocol, \kanon{}, and the analysis of its search accuracy and susceptibility to certain inversion attacks.

	The key take-away and future research directions that this thesis highlights are fourfold.

	\subsubsection{Practicality and reproducibility}

		First and foremost, future research in the area of secure outsourced database systems should focus more prominently on practicality and reproducibility.
		After analyzing a plethora of works in the literature (see \cite{ore-benchmark-17,epsolute}) we discovered that a large fraction of constructions either do not have experiments, or their code is unavailable or otherwise not suitable for inspection, or the experimental results are not reproducible.
		We firmly believe in the reproducibility mission (such as SIGMOD\footnote{\url{https://reproducibility.sigmod.org}} and pVLDB\footnote{\url{https://vldb.org/pvldb/reproducibility/}} efforts) and we encourage the works in the area to join the initiative.\footnote{
			We note that our work \cite{ore-benchmark-17} has received ``Most Reproducible Paper'' award.
		}

	\subsubsection{Practicality of property-preserving encryption}

		Second, our works \cite{ore-benchmark-17,k-anon} demonstrate the practical value of property-preserving encryption as a component of a secure database system.
		While an argument can be made that a property-preserving encryption is inherently less-than-ideally secure from a purely cryptographic perspective, we counter that its value is much greater in a practical outsourced database system, which may not necessarily require perfect secrecy.
		A construction using such encryption scheme can be practical as long as the scheme's performance is measured, its leakage is quantified and the effect of this leakage on the security of the entire system is properly analyzed.

	\subsubsection{Practicality of using ``heavy'' primitives and protocols}

		Third, as our work, \epsolute{}, demonstrates, the primitives and protocols that are (rightly) considered heavyweight, such as \acrshort{oram} and \acrshort{dp}-sanitizers, can still be used efficiently in an outsourced system.
		In \epsolute{}, we show that a clever parallelization and optimization on both macro and micro levels can result in a very fast system overall.\footnote{
			We have independently explored running \epsolute{} in a \acrlong{tee}, and we have observed even higher performance.
		}
		We encourage practitioners to revisit using ``heavy'' primitives and protocols, such as \acrshort{oram}, homomorphic encryption, garbled circuits, zero-knowledge proofs, in their systems.

	\subsubsection{More query types}

		Finally, while we have covered three query types for a secure outsourced database system, we need more types to build a full-featured database that can compete with existing mainstream \acrshort{rdbms} like PostgreSQL\@.
		The directions include \texttt{JOIN}, \texttt{GROUP BY}, \texttt{AGGREGATE} queries and custom predicates.
