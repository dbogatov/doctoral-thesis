\chapter{Background}\label{section:background}
\thispagestyle{myheadings}

	In this section, I will go over the building blocks required to construct the outsourced database systems and their components that I will discuss in the next chapters.
	These prerequisites include the symmetric encryption, \acrshortpl{oram} and PathORAM \cite{path-oram} in particular, \acrlong{dp} and \acrshort{dp} sanitizers, and finally, \acrlongpl{tee}.
	Some of the following sections were paraphrased or taken verbatim from my published work, \cite{ore-benchmark-17,epsolute}.

	\section{Symmetric encryption}

	\section{\texorpdfstring{\acrlong{oram}}{Oblivious Random Access Machine}}

		\subsection{PathORAM}

	\section{\texorpdfstring{\acrlong{dp}}{Differential Privacy}}

		The most effective approach is sampling noise using \acrfull{dp}.
		\acrshort{dp} is a guarantee on a query algorithm that takes a database and returns some result.
		The guarantee states that for two neighboring databases (that differ in exactly one record), the probability that the adversary will understand by looking at the output, which of the two databases was used as an input, is bounded.
		More formally, the \acrlong{dp} is defined in \cref{definition:dp}.

		\begin{definition}[\acrlong{dp}, adapted from \cite{our-data-ourselves, differential-privacy-original}]\label{definition:dp}
			A randomized algorithm \algo{A} is $(\epsilon, \delta)$-differentially private if for all $\database_1 \sim \database_2 \in \searchKeyDomain^\dataSize$, and for all subsets $\mathcal{O}$ of the output space of \algo{A},
			\[
				\probability{ \algo{A}{ \database_1 } \in \mathcal{O} } \leq \exp(\epsilon) \cdot \probability{ \algo{A}{ \database_2 } \in \mathcal{O} } + \delta \; .
			\]
		\end{definition}

		One way to interpret this definition is the following.
		Probabilities are taken over the coins of algorithm \algo{A}, which answers a query based on a dataset.
		A natural instantiation of \algo{A} is a view of a distinguishing adversary \adversary{}, who tries to guess which of the two datasets was used.
		The expression in \cref{definition:dp} then bounds the advantage of \adversary{} with $\epsilon$ and $\delta$ parameters.
		Note that $\exp( x ) \approx 1 + x + \frac{x^2}{2!}$ and for sufficiently small $x$ the last term is negligible.
		If we put $\epsilon + 1$ in place of $\exp( \epsilon )$, it becomes clear that $\epsilon$ is the exact value by which two probabilities are allowed to differ.
		For $\epsilon = 0$, they have to be equal, for $\epsilon = 0.01$, probabilities may differ by \SI{1}{\percent}.
		Therefore, $\epsilon$ is called \emph{a privacy budget} of a \acrshort{dp} system.
		$\delta$ term is additive and therefore must be small by itself.
		This term is essentially a probability that the entire system fails.
		For example, if \algo{A} is a randomized algorithm that fails with a certain chance, this probability will be $\delta$.
		For instance, a PathORAM \cite{path-oram} algorithm can have a stash overflow with a bounded probability \cite[Theorem 1]{path-oram} and it will cause the entire algorithm to fail.
		If PathORAM is used in a \acrshort{dp} system then this probability, however small, bounded and negligible, will have to be accounted for in $\delta$.

		Note that \cref{definition:dp} describes a property of \algo{A} and not a construction method.
		To construct \algo{A}, the seminal \cite{differential-privacy-original} offers an algorithm called \acrfull{lpa}.
		The idea is to tune the noise sampled from the Laplacian distribution to the \emph{sensitivity} of a query, defined as the change of output with respect to change in input.
		For example, if a change in one record of the dataset causes a change in the output value of at most one (e.g., a count query), then the sensitivity is 1.
		\cite{differential-privacy-original} proves that if one adds $\algo{Laplace}{0, \frac{\mathsf{sensitivity}}{\epsilon}}$ to the real result of a query, the resulting mechanism is $\epsilon$-\acrshort{dp}.

		\subsection{\texorpdfstring{\acrshort{dp}}{DP} sanitizers}

			Lastly, instead of generating and adding noise each query, it is possible to generate noisy values once, embed them into the data and release the sanitized data without compromising privacy guarantees.
			The intuition is that once the noise is embedded in the dataset, the adversary can run a regular query over it and the entire query will still be \acrshort{dp}.
			Such methods are collectively called \emph{sanitization}, and there are known bounds for point and range queries, for pure ($\delta = 0$) and approximate ($\delta > 0$) \acrshort{dp} \cite{bounds-on-sample-complexity,private-learning-and-sanitization,non-interactive-database-privacy,dp-under-observation,dp-release,privately-learning-thresholds}.

		\section{\texorpdfstring{\acrlongpl{tee}}{Trusted Execution Environments}}
