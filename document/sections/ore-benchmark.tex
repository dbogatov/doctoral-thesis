\chapter{Range queries in a snapshot model}\label{section:range-queries-snapshot}
\thispagestyle{myheadings}

	One of the ways to run a range query in a presence of a snapshot adversary is to use a conventional database range index, such as \BPlus{} tree~\cite{b-tree}, and encrypt the values in the index in a way that preserves their comparison result.
	This is exactly what \gls{ope} does.
	\gls{ope} is a scheme, a tuple of algorithms \algo{OPE.KeyGen}, \algo{OPE.Enc} and \algo{OPE.Dec}, that generate a kay, encrypt and decrypt a number respectively with a property that if $x$ was smaller, greater or equal to $y$, then their respective ciphertexts will maintain the relation.
	\gls{ore} works similarly except that the ciphertexts are not necessarily numbers and a comparison is an explicit algorithm \algo{ORE.Cmp} over hte ciphertexts.
	Formally, \gls{ope} is a specific \gls{ore} where ciphertexts are numbers and comparison is trivial, therefore, in this work I will only refer to \gls{ore}.

	Security of an \gls{ore} scheme is typically defined as via security game and a leakage profile \cite{practical-ore}.
	The scheme is defined secure with a leakage \leakage{} if there exists a simulator that can use the leakage function and can generate output indistinguishable from the one generated by the real scheme \cite{ore-benchmark-17}.
	The leakage function, ranging from as much as half of the bits of input to as little as an equality pattern of the most-significant differing bit of two inputs, is the defining security level of a scheme.

	In the comparative evaluation work \cite{ore-benchmark-17}, I choose five \gls{ore} scheme and analyze them under common framework.
	For each of the schemes I identify how many times it uses which cryptographic or expensive algebraic primitives, such as PRG or samplers, its ciphertext or state size and, its leakage profile.
	I deliberately do not account wall-clock execution time as this value is highly dependent on hardware and primitive implementations and is, therefore, not representative.
	Below I concisely describe the schemes and give a summary of my finding in \cref{table:ore}.

	% cSpell:ignore Lewi BCLO CLWW captionsetup

\begin{sidewaystable}
	\renewcommand{\arraystretch}{1.5}
	\centering
	\captionsetup{width=\textwidth}
	\caption[Primitive usage by OPE / ORE schemes]{
		\cite[Tables 1 and 4]{ore-benchmark-17}.
		Primitive usage by OPE / ORE schemes.
		Ordered by security rank --- most secure below.
		$n$ is the input length in bits, $d$ is a block size for Lewi-Wu~\cite{lewi-wu-ore} scheme, $\lambda$ is a PRF output size, $N$ is a total data size, \textbf{HG} is a hyper-geometric distribution sampler, \textbf{PPH} is a property-preserving hash with $h$-bit outputs built with bilinear maps and \textbf{bolded} are weak points of the schemes.
		Values in parentheses are simulation-derived. $N = 10^3$, $n = 32$, $d = 2$, $\lambda = 128$ and $h = 128$ in this simulation.
	}\label{table:ore}
	\begin{tabular*}{\linewidth}{ !{\extracolsep\fill} l c c c c c } % chktex 26

		\toprule

		\multirow{2}{*}{Scheme}						& \multicolumn{2}{c}{Primitive usage (number of invocations)}														& \multirow{2}{*}{\makecell{Ciphertext size, \\ or state size (bits)}}					&  \multirow{2}{*}{\makecell{Leakage \\ (in addition to inherent total order)}}				\\ \cline{2-3}
		\rule{0pt}{10pt}							& Encryption														& Comparison									& 																						& 																							\\

		\toprule

		\cite{bclo-ope}								& $\bm{n}$ (41) \textbf{HG}											& none											& $2n$ (64)																				& \textbf{$\approx$ Top half of the bits}													\\

		\midrule

		\cite{clww-ore}								& $n$ (32) PRF 														& none											& $2n$ (64)																				& \textbf{Most-significant differing bit}													\\

		\midrule

		\multirow{3}{*}{\cite{lewi-wu-ore}}			& \boldmath{} $\nicefrac{2n}{d}$ \unboldmath{} (32) \textbf{PRP}	& \multirow{3}{*}{$\frac{n}{2d}$ (9) Hash}		& 																						& \multirow{3}{*}{Most-significant differing block}											\\
													& $2 \frac{n}{d} \left( 2^d + 1 \right)$ (160) PRF					&												& $\frac{n}{d} \left(\lambda + n + 2^{d + 1} \right) + \lambda$							&																							\\
													& $\frac{n}{d} 2^d$ (64) Hash										&												& (2816)																				&																							\\


		\midrule

		\multirow{3}{*}{\cite{cloz-ore}}			& $n$ (32) PRF													& \multirow{3}{*}{$\bm{n^2}$ (1046) \textbf{PPH}}	& \multirow{3}{*}{$n \cdot h$ (4096)}													& \multirow{3}{*}{\makecell{Equality pattern \\ of the most-significant \\ differing bit}}	\\
													& $n$ (32) PPH													&													&																						& 																							\\
													& 1 PRP															&													&																						& 																							\\

		\midrule

		\cite{fh-ope}								& 1 Traversal													& 3 Traversals										& $\bm{3 \cdot n \cdot N}$ (86842)														& Insertion order																			\\

		\bottomrule

	\end{tabular*}
\end{sidewaystable}


