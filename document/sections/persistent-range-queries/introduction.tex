\section{Introduction}

	Secure outsourced database systems aim at helping organizations outsource their data to untrusted third parties, without compromising data confidentiality or query efficiency.
	The main idea is to encrypt the data records before uploading them to an untrusted server along with an index data structure that governs which encrypted records to retrieve for each query.
	While strong cryptographic tools can be used for this task, existing implementations such as CryptDB \cite{crypt-db}, Cipherbase \cite{cipherbase-daas}, StealthDB \cite{stealth-db} and TrustedDB \cite{trusted-db} try to optimize performance but do not provide strong security guarantees when answering queries.
	Indeed, a series of works \cite{multidimensional-range-queries, inference-attack-islam-14, leakage-abuse-attacks-cash-15, inference-attacks-naveed-15, generic-attacks-kellaris, attacks-tao-of-inference, grubbs-attacks, access-pattern-disclosure, attacks-improved-reconstruction} demonstrate that these systems are vulnerable to a variety of reconstruction attacks.
	That is, an adversary can fully reconstruct the distribution of the records over the domain of the indexed attribute.
	This weakness is prominently due to the \emph{access pattern leakage}: the adversary can tell if the same encrypted record is returned on different queries.

	More recently, \cite{generic-attacks-kellaris, state-of-uniform, attacks-improved-reconstruction, pump-volume-attacks, volume-range-attacks} showed that reconstruction attacks are possible even if the systems employ heavyweight cryptographic techniques that hide the access patterns, such as homomorphic encryption \cite{arbitrary-functions-encrypted, fully-homomorphic-encryption} or \acrfull{oram} \cite{oram-theory, oram-original}, because they leak the size of the result set of a query to the server (this is referred to as \emph{communication volume leakage}). % chktex 2
	Thus, even some recent systems that provide stronger security guarantees like ObliDB \cite{oblidb}, Opaque \cite{opaque} and Oblix \cite{oblix} are susceptible to these attacks. % chktex 2
	This also means that no outsourced database system can be both optimally efficient and privacy-preserving: secure outsourced database systems should not return the exact number of records required to answer a query.

	We take the next step towards designing secure outsourced database systems by presenting novel constructions that strike a provable balance between efficiency and privacy.
	First, to combat the access pattern leakage, we integrate a layer of \acrshort{oram} storage in our construction.
	Then, we bound the communication volume leakage by utilizing the notion of \acrfull{dp} \cite{differential-privacy-original}.
	Specifically, instead of returning the exact number of records per query, we only reveal perturbed query answer sizes by adding random encrypted records to the result so that the communication volume leakage is bounded.
	Our construction guarantees privacy of any single record in the database which is necessary in datasets with stringent privacy requirements.
	In a medical \acrshort{hipaa}-compliant setting, for example, disclosing that a patient exists in a database with a rare diagnosis correlating with age may be enough to reveal a particular individual.

	The resulting mechanism achieves the required level of privacy, but implemented na\"{\i}vely the construction is prohibitively slow.
	We make the solution practical by limiting the amount of noise and the number of network roundtrips while preserving the privacy guarantees.
	We go further and present a way to parallelize the construction, which requires adapting noise-generation algorithms to maintain differential privacy requirements.

	Using our system, we have run an extensive set of experiments over cloud machines, utilizing large datasets --- that range up to 10 million records --- and queries of different sizes, and we report our experimental results on efficiency and scalability.
	We compare against best possible solutions in terms of efficiency (conventional non-secure outsourced database systems on unencrypted data) and against an approach that provides optimal security (retrieves the full table from the cloud or runs the entire query obliviously with maximal padding).
	We report that our solution is very competitive against both baselines.
	Our performance is comparable to that of unsecured plain-text optimized database systems (like MySQL and PostgreSQL): while providing strong security and privacy guarantees, we are only 4 to 8 times slower in a typical setting.
	Compared with the optimally secure solution, a linear scan (downloading all the records), we are 18 times faster in a typical setting and even faster as database sizes scale up.

	\smallskip

	To summarize, our contributions in this work are as follows:
	\begin{itemize}
		\item
			We present a new model for a \emph{differentially private} outsourced database system, \acrshort{cdpodb}, its security definition, query types, and efficiency measures.
			In our model, the adversarial honest-but-curious server cannot see the record values, access patterns, or exact communication volume.

		\item
			We describe a novel construction, \epsolute{}, that satisfies the proposed security definition, and provide detailed algorithms for both range and point query types.
			In particular, to conceal the access pattern and communication volume leakages, we provide a secure storage construction, utilizing a combination of \acrlong{oram} \cite{oram-theory, oram-original} and differentially private sanitization \cite{non-interactive-database-privacy}.
			Towards this, we maintain an index structure to know how many and which objects we need to retrieve.
			This index can be stored locally for better efficiency (in all our experiments this is the case), but crucially, it can also be outsourced to the adversarial server and retrieved on-the-fly for each query.

		\item
			We improve our generic construction to enable parallelization within a query.
			The core idea is to split the storage among multiple \acrshortpl{oram}, but this requires tailoring the overhead required for differential privacy proportionally to the number of \acrshortpl{oram}, in order to ensure privacy.
			We present practical improvements and optimization techniques that dramatically reduce the amount of fetched noise and the number of network roundtrips.

		\item
			Finally, we provide and open-source a high-quality C++ implementation of our system.
			We have run an extensive set of experiments on both synthetic and real datasets to empirically assess the efficiency of our construction and the impact of our improvements.
			We compare our solutions to the na\"{\i}ve approach (linear scan downloading all data every query), oblivious processing and maximal padding solution (Shrinkwrap \cite{shrinkwrap}), and to a non-secure regular \acrshort{rdbms} (PostgreSQL and MySQL), and we show that our system is very competitive.
	\end{itemize}

	\subsection{Related Work}

		We group the related secure databases, engines, and indices into three categories
		\begin{enumerate*}[label={(\roman*)}]
			\item systems that are oblivious or volume-hiding and do not require \acrfull{tee},
			\item constructions that rely on \acrshort{tee} (usually, Intel \acrshort{sgx}),
			\item solutions that use property-preserving or semantically secure encryption and target primarily a snapshot adversary.
		\end{enumerate*}
		\emph{We claim that \epsolute{} is the most secure and practical range- and point-query engine in the outsourced database model, that protects both \acrfull{ap} and \acrfull{cv} using \acrlong{dp}, while not relying on \acrshort{tee}, linear scan or padding result size to the maximum.}

		\paragraph*{Obliviousness and volume-hiding without enclave}

			This category is the most relevant to \epsolute{}, wherein the systems provide either or both \acrshort{ap} and \acrshort{cv} protection without relying on \acrshort{tee}.
			\crypte{} \cite{crypte} is a recent end-to-end system executing ``\acrshort{dp} programs''.
			\crypte{} has a different model than \epsolute{} in that it assumes two non-colluding servers, an adversarial querying user (the analyst), and it uses \acrshort{dp} to protect the privacy of an individual in the database, which includes volume-hiding for aggregate queries.
			\crypte{} also does not consider oblivious execution and attacks against the \acrshort{ap}.
			Shrinkwrap \cite{shrinkwrap} (and its predecessor SMCQL \cite{smcql}) is an excellent system designed for complex queries over federated and distributed data sources.
			In Shrinkwrap, \acrshort{ap} protection is achieved by using oblivious operators (linear scan and sort) and \acrshort{cv} is concealed by adding fake records to intermediate results with \acrshort{dp}.
			Padding the result to the maximum size first and doing a linear scan over it afterwards to ``shrink'' it using \acrshort{dp}, is much more expensive than in \epsolute{}, however.
			In addition, in processing a query, the worker nodes are performing an $O(n \log{n})$ cost oblivious sorting, where $n$ is the maximum result size (whole table for range query), since they are designed to answer more general complex queries.
			SEAL \cite{seal} offers adjustable \acrshort{ap} and \acrshort{cv} leakages, up to specific bits of leakage.
			SEAL builds on top of Logarithmic-SRC \cite{practical-range-search}, splits storage into multiple \acrshortpl{oram} to adjust \acrshort{ap}, and pads results size to a power of 2 to adjust \acrshort{cv}.
			\epsolute{}, on the other hand, fully hides the \acrshort{ap} and uses \acrshort{dp} with its guarantees to pad the result size.
			PINED-RQ \cite{pined-rq} samples Laplacian noise right in the B+ tree index tree, adding fake and removing real pointers according to the sample.
			Unlike \epsolute{}, PINED-RQ allows false negatives (i.e., result records not included in the answer), and does not protect against \acrshort{ap} leakage.
			On the theoretical side, \textcite{differential-obliviousness} (followed by \textcite{differential-obliviousness-followup}) treat the \acrshort{ap} itself as something to protect with \acrshort{dp}.
			\cite{differential-obliviousness} introduces a notion of differential obliviousness that is admittedly weaker than the full obliviousness used in \epsolute{}. % chktex 2
			Most importantly, \cite{differential-obliviousness} ensures differential privacy w.r.t.~the \acrshort{oram} only, while \epsolute{} ensures \acrshort{dp} w.r.t.~the entire view of the adversary. % chktex 2

		\paragraph*{Enclave-based solutions}

			Works in this category use trusted execution environment (usually, \acrshort{sgx} enclave).
			These works are primarily concerned with the \acrshort{ap} protection for both trusted and untrusted memory, unlike \epsolute{} which also protects \acrshort{cv}.
			Cipherbase \cite{cipherbase-daas} was a pioneer introducing the idea of using \acrshort{tee} (\acrshort{fpga} at that time) to assist with \acrshort{rdbms} security.
			HardIDX \cite{hardidx} simply puts the B+ tree in the enclave, while StealthDB \cite{stealth-db} symmetrically encrypts all records and brings them in the enclave one at a time for processing.
			EnclaveDB \cite{enclave-db} assumes somewhat unrealistic \SI{192}{\giga\byte} enclave and puts the entire database in it.
			ObliDB \cite{oblidb} and Opaque \cite{opaque} assume fully oblivious enclave memory (not available as of today) and devise algorithms that use this fully trusted portion to obliviously execute common \acrshort{rdbms} operators, like filters and joins.
			Oblix \cite{oblix} provides a multimap that is oblivious both in and out of the enclave.
			HybrIDX claims protection against both \acrshort{ap} and \acrshort{cv} leakages, but unlike \epsolute{} it only obfuscates them.
			\epsolute{} offers an indistinguishability guarantee for \acrshort{ap} and a \acrshort{dp} guarantee for \acrshort{cv}, while HybrIDX hides the exact result size and only obfuscates the \acrshort{ap}.
			Lastly, Hermetic \cite{hermetic} takes on the \acrshort{sgx} side-channel attacks, including \acrshort{ap}.
			It provides oblivious primitives, however, it only offers protection against software and not physical attacks (e.g., it trusts a hypervisor to disable interrupts).

		\paragraph*{Solutions against the snapshot adversary}

			Works in this category protect against the snapshot adversary, which takes a snapshot of the data at a fixed point in time (e.g., stolen hard drive).
			We stress that \epsolute{} provides semantic security against the snapshot adversary on top of \acrshort{ap} and \acrshort{cv} protection.
			CryptDB \cite{crypt-db} is a seminal work in this direction offering computations over encrypted data.
			It has since been shown (e.g. \cite{inference-attacks-naveed-15,inference-attack-islam-14,attacks-tao-of-inference}) that the underlying property-preserving schemes allow for reconstruction attacks.
			Arx \cite{arx} provides strictly stronger security guarantees by using only semantically secure primitives.
			Seabed \cite{seabed} uses an additively symmetric homomorphic encryption scheme for aggregates and certain filter queries.
			\textcite{ppqed} offer a method to verify and apply a predicate (a junction of conditions) using garbled circuits or homomorphic encryption without revealing the predicate itself.
			SisoSPIR \cite{sisospir} presents a mechanism to build an oblivious index tree such that neither party learns the pass taken.
			See \cite{ore-benchmark-17} for a survey of range query protocols in this category. % chktex 2
