\section{Differentially private outsourced database systems}\label{section:range-persistent:dpodb}

	In this section we present our model, \emph{differentially private outsourced database system}, \acrshort{cdpodb}, its security definition, query types and efficiency measures.
	It is an extension of the outsourced database model in \cref{section:range-persistent:background}.

	\subsection{Adversarial model}\label{section:range-persistent:dpodb:adversarial-models}

		We consider an honest-but-curious polynomial time adversary that attempts to breach differential privacy with respect to the input database \database{}.
		We observe later in \cref{section:range-persistent:dpodb:adversarial-models:adaptive} that it is impossible to completely hide the number of records returned on each query without essentially returning all the database records on each query.
		This, in turn, means that different query sequences may be distinguished, and, furthermore, that differential privacy may not be preserved if the query sequence depends on the content of the database records.
		We hence, only require the protection of differential privacy with respect to every fixed query sequence.
		Furthermore, we relax to computational differential privacy (following~\cite{computational-dp}).

		In the following definition, the notation \view{\protocol{}}{\database, \fromNtoM{\query}{1}{m}} denotes the view of the server \server{} in the execution of protocol \protocol{} in answering queries \fromNtoM{\query}{1}{m} with the underlying database \database{}.

		\begin{definition}
			We say that an outsourced database system \protocol{} is $(\epsilon, \delta)$-computationally differentially private (a.k.a.~\acrshort{cdpodb}) if for every polynomial time distinguishing adversary \adversary{}, for every neighboring databases $\database \sim \database^\prime$, and for every query sequence $\fromNtoM{\query}{1}{m} \in \querySet^m$ where $m = \mathsf{poly}(\lambda)$,

			\begin{multline*}
				\probability{\adversary \left( 1^\lambda, \view{\protocol{}}{\database, \fromNtoM{\query}{1}{m}} \right) = 1 } \leq \\
				\exp{\epsilon} \cdot \probability{\adversary \left( 1^\lambda, \view{\protocol{}}{\database^\prime, \fromNtoM{\query}{1}{m}} \right) = 1} + \delta +\negl \; ,
			\end{multline*}
			where the probability is over the randomness of the distinguishing adversary \adversary{} and the protocol \protocol{}.
		\end{definition}

	\begin{remark}[Informal]
		We note that security and differential privacy in this model imply protection against communication volume and access pattern leakages and thus prevent a range of attacks, such as \cite{leakage-abuse-attacks-cash-15,inference-attacks-naveed-15,generic-attacks-kellaris}. % chktex 2
	\end{remark}

	\subsubsection{On impossibility of adaptive queries}\label{section:range-persistent:dpodb:adversarial-models:adaptive}

		Non-adaptivity in our \acrshort{cdpodb} definition does not reflect a deficiency of our specific protocol but rather an inherent source of leakage when the queries may depend on the decrypted data.
		Consider an adaptive \acrshort{cdpodb} definition that does not fix the query sequence \fromNtoM{q}{1}{m} in advance but instead an arbitrary (efficient) user \user{} chooses them during the protocol execution with \server{}.
		As before, we ask that the \server{}'s view is \acrshort{dp} on neighboring databases for every such \user{}.
		We observe that this definition cannot possibly be satisfied by \emph{any} outsourced database system without unacceptable efficiency overhead.
		Note that non-adaptivity here does not imply that the client knows all the queries in advance, but rather can choose them at any time (e.g., depending on external circumstances) as long as they do not depend on true answers to prior queries.

		To see this, consider two neighboring databases $\database, \database^\prime$.
		Database \database{} has 1 record with $\mathsf{key} = 0$ and $\database^\prime$ has none.
		Furthermore, both have 50 records with $\mathsf{key} = 50$ and 100 records with $\mathsf{key} = 100$.
		User \user{} queries first for the records with $\mathsf{key} = 0$, and then if there is a record with $\mathsf{key} = 0$ it queries for the records with $\mathsf{key} = 50$, otherwise for the records with $\mathsf{key} = 100$.
		Clearly, an efficient outsourced database system cannot return nearly as many records when $\mathsf{key} = 50$ versus $\mathsf{key} = 100$ here.
		Hence, this allows distinguishing $\database, \database^\prime$ with probability almost 1.

		To give a concrete scenario, suppose neighboring medical databases differ in one record with a rare diagnosis ``Alzheimer's disease''.
		A medical professional queries the database for that diagnosis first (point query), and if there is a record, she queries the senior patients next (range query, \texttt{age $\ge$ 65}), otherwise she queries the general population (resulting in more records).
		We leave it open to meaningfully strengthen our definition while avoiding such impossibility results, and we defer the formal proof to future work.

	% chktex-file 1
% chktex-file 8
% chktex-file 21
% chktex-file 24
% chktex-file 26
% chktex-file 36
% chktex-file 37

\newlength{\setupLength}
\setlength{\setupLength}{17em}
\newlength{\queryLength}
\setlength{\queryLength}{16em}

\begin{algorithm*}[ht!]

	\begin{pcvstack}

		\procedure[linenumbering]{\protocolSetup{}}{
													\textbf{User \user}																\>																\> \textbf{Server \server}	\\
			%
			\label{algorithm:dp-oram:setup:line-2}	\pcinput{\database}																\>																\> \pcinput{\emptyset}		\\
			%
			\label{algorithm:dp-oram:setup:line-3}	\indexI \gets \algo{CreateIndex}{\database}										\>																\>							\\
			%
			\label{algorithm:dp-oram:setup:line-4}	\oramProgram = \left. (\oramWrite, \recordID_i, \record_i) \right|_{i = 1}^n	\>																\>							\\
			%
			\label{algorithm:dp-oram:setup:line-5}																					\> \sendmessageboth*[\setupLength]{\algo{ORAM}{\oramProgram}}	\>							\\
			%
			\label{algorithm:dp-oram:setup:line-6}	\serverDS \gets \algo{A}{\fromNtoM{\searchKey}{1}{\domainSize}}					\> \sendmessageright*[\setupLength]{\serverDS}					\>							\\
			%
			\label{algorithm:dp-oram:setup:line-7}	\pcouput{\indexI}																\>																\> \pcouput{\serverDS}
		}

		\vspace{0.5em}

		\procedure[linenumbering]{\protocolQuery{}}{
													\textbf{User \user}																				\>																											\> \textbf{Server \server}				\\
			%
			\label{algorithm:dp-oram:query:line-2}	\pcinput{\query, \indexI}																		\>																											\> \pcinput{\serverDS}					\\
			%
			\label{algorithm:dp-oram:query:line-3}	T \gets \algo{Lookup}{\indexI, \query}															\> \sendmessageright*[\queryLength]{\query}																	\> c \gets \algo{B}{\serverDS, \query}	\\
			%
			\label{algorithm:dp-oram:query:line-4}	\oramProgram_\mathsf{true} = \left. (\oramRead, \recordID_i, \bot) \right|_{i \in T}			\> \sendmessageleft*[\queryLength]{c}																		\>										\\
			%
			\label{algorithm:dp-oram:query:line-5}	\oramProgram_\mathsf{noise} = \left. (\oramRead, S \setminus T, \bot) \right|_{1}^{c - \abs{T}}	\>																											\>										\\
			%
			\label{algorithm:dp-oram:query:line-6}	R																								\> \sendmessageboth*[\queryLength]{\algo{ORAM}{\oramProgram_\mathsf{true} \| \oramProgram_\mathsf{noise}}}	\>										\\
			%
			\label{algorithm:dp-oram:query:line-7}	\pcouput{R}																						\>																											\>	\pcouput{\emptyset}
		}

	\end{pcvstack}

	\caption[\epsolute{} protocol]{
		\epsolute{} protocol.
		$\algo{ORAM}{\cdot}$ denotes an execution of \acrshort{oram} protocol (\cref{section:background:oram}), where \user{} plays the role of the client.
		\acrshort{oram} protocol client and server states are implicit.
		$S \setminus T$ represents a set of valid record IDs $S$ that are not in the true result set $T$.
	}%
	\label{algorithm:dp-oram}
\end{algorithm*}


	\subsection{Query types}

		In this work we are concerned with the following query types:
		\begin{description}[style=unboxed, leftmargin=0em]
			\item[Range queries]
				Here we assume a total ordering on \searchKeyDomain{}.
				A query \query{a}{b} is associated with an interval $\interval{a}{b}$ for $1 \leq a \leq b \leq \domainSize$ such that $\query{a}{b}(c) = 1$ iff $c \in \interval{a}{b}$ for all $c \in \searchKeyDomain$.
				The equivalent \acrshort{sql} query is:

				\smallskip
				\indent\texttt{SELECT * FROM table WHERE attribute BETWEEN a AND b;}
				\smallskip

			\item[Point queries]
				Here \searchKeyDomain{} is arbitrary and a query predicate \query{a} is associated with an element $a \in \searchKeyDomain$ such that $\query{a}(b) = 1$ iff $a = b$.
				In an ordered domain, point queries are degenerate range queries.
				The equivalent \acrshort{sql} query is:

				\smallskip
				\indent\texttt{SELECT * FROM table WHERE attribute = a;}

		\end{description}

	\subsection{Measuring Efficiency}

		We define two basic efficiency measures for a \acrshort{cdpodb}.
		\begin{description}[style=unboxed, leftmargin=0em]
			\item[Storage efficiency]
				is defined as the sum of the bit-lengths of the records in a database relative to the bit-length of a corresponding encrypted database.
				Specifically, we say that an outsourced database system has \emph{storage efficiency} of $(\efficiencyCoefficient, \efficiencyOffset)$ if the following holds.
				Fix any \databaseDef{} and let $n_1 = \sum_{i=1}^n \abs{r_i}$.
				Let $\server_\mathsf{state}$ be an output of \server{} on a run of \protocolSetup{} where \user{} has input \database{}, and let $n_2 = \abs{ \server_\mathsf{state} }$.
				Then $n_2 \leq \efficiencyCoefficient n_1 + \efficiencyOffset$.

			\item[Communication efficiency]
				is defined as the sum of the lengths of the records in bits whose search keys satisfy the query relative to the actual number of bits sent back as the result of a query.
				Specifically, we say that an outsourced database system has \emph{communication efficiency} of $(\efficiencyCoefficient, \efficiencyOffset)$ if the following holds.
				Fix any \query{} and \serverDS{} output by \protocolSetup{}, let \user{} and \server{} execute \protocolQuery{} where \user{} has inputs \query{}, and output $R$, and \server{} has input \serverDS{}.
				Let $m_1$ be the amount of data in bits transferred between \user{} and \server{} during the execution of \protocolQuery{}, and let $m_2 = \abs{ R }$.
				Then $m_2 \leq \efficiencyCoefficient m_1 + \efficiencyOffset$.
		\end{description}

		Note that $\efficiencyCoefficient \geq 1$ and $\efficiencyOffset \geq 0$ for both measures.
		We say that an outsourced database system is \emph{optimally storage efficient} (resp., \emph{optimally communication efficient}) if it has storage (resp., communication) efficiency of $(1, 0)$.
