\subsection{FH-OPE}

	Frequency-hiding \acrshort{ope} by \textcite{fh-ope} is a stateful scheme that hides the frequency of the plaintexts, so the adversary is unable to construct a frequency histogram.

	This scheme is stateful, which means that the client needs to keep a data structure and update it with every encryption and decryption.
	The data structure is a binary search tree where the node's value is the plaintext and node's position in a tree is the ciphertext.
	For example, consider the range $[1, 128]$.
	Any plaintext that happens to arrive first (for example, $6$), will be the root, and thus the ciphertext is $64$.
	Then any plaintext smaller than the root, say $3$, will become the left child of the root, and will produce the ciphertext $32$.
	To encrypt a value, the algorithm traverses the tree until it finds a spot for the new plaintext, or finds the same plaintext.
	If the same plaintext is found, the traversal pseudo-randomly passes to the left or right child, up to the leaf.
	This way, the invariant of the tree --- intervals of the same plaintexts do not overlap --- is maintained.
	The ciphertext generated from the new node's position is returned.

	Due to randomized ciphertexts, the comparison algorithm is more complicated than in the regular deterministic \acrshort{ope}.
	To properly compare ciphertexts, the algorithm needs to know the boundaries --- the minimum and maximum ciphertexts for a particular plaintext.
	The client is responsible for traversing the tree to find the plaintext for the ciphertext and then minimum and maximum ciphertext values.
	Having these values, the comparison is trivial --- equality is a check that the value is within the boundaries, and other comparison operators are similar.

	Authors have designed a number of heuristics to minimize the state size, however, these are mostly about compacting the tree and the result depends highly on the tree content.
	In our analysis, we consider the worst case performance without the use of heuristics.
	In our experimental evaluation, however, we did implement compaction.

	\subsubsection{Security}
		The security of the scheme relies on the large range size to domain size ratio.
		Authors recommend at least 6 times longer ciphertexts than the plaintexts in bit-length, which means ciphertexts should be 192-bit numbers that are not commonly supported.
		It is possible to operate over arbitrary-length numbers, but the performance overhead would be substantial.
		We did a quick micro-benchmark in {\Csharp} and the overhead of using \texttt{BigInteger} is 15--20 times for basic arithmetic operations.

		This scheme satisfies IND-FAOCPA definition (introduced along with the scheme \cite{fh-ope}), meaning that it does not leak the equality or relative distance between the plaintexts.
		This definition has been criticized in \cite{florian-def-critique}, who claim that the definition is imprecise and propose an enhanced definition along with a small change to construction to satisfy this new definition.
		Both schemes leak the insertion order, because it affects the tree structure.
		We do not know of any attacks against this leakage, but it does not mean they cannot exist.
		\textcite{leakage-abuse-grubs-2017} describe an attack against this scheme (binomial attack), but it applies to any perfectly secure (leaking only total order) frequency-hiding \acrshort{ope}.

	\subsubsection{Analysis and implementation challenges}

		If the binary tree grows in only one direction, at some point it will be impossible to generate another ciphertext.
		In this case, the tree has to be rebalanced.
		This procedure will invalidate all ciphertexts already generated.
		This property makes the scheme difficult to use in some protocols since they usually rely on the ciphertexts on the server being always valid.
		The authors explicitly mention that the scheme works under the assumption of uniform input.
		However, the rebalancing will be caused by insertion of just 65 consecutive input elements for 64-bit integer range.

		The scheme makes one tree traversal on encryption and decryption.
		Comparison is trickier as it requires one traversal to get the plaintext, and two traversals for minimum and maximum ciphertexts.
		We understand that it is possible to get these values in fewer than three traversals, but we did not optimize the scheme for the analysis and evaluation.

		For practitioners we note that the stateful nature of the scheme implies that the client storage is no longer negligible as the state grows proportionally to the number of encryptions.
		We also note that implementing compaction extensions will affect code complexity and performance.
		Finally, we stress again that some non-uniform inputs can break the scheme by causing all ciphertexts to be invalid.
		It is up to the users of the scheme to ensure uniformity of the input, which poses serious restrictions on the usage of the scheme.
