\subsection{Kerschbaum-Tueno}

	\textcite{florian-protocol} proposed a new data structure, which satisfies their own definitions of security (IND-CPA-DS) and efficiency (search operation has poly\hyp{}logarithmic running time and linear space complexity).

	In short, the idea is to maintain a (circular) array of symmetrically encrypted ciphertexts in order.
	On insertion, the array is rotated around a uniformly sampled offset to hide the location of the smallest element.
	Client interactively performs a binary search requesting an element, decrypting it and deciding which way to go.

	\subsubsection{Security}

		Authors prove that this construction is IND-CPA-DS secure (defined in the same paper \cite{florian-protocol}).
		The definition assumes an array data structure and therefore serves specifically this construction (as opposed to being generic).
		It provably hides the frequency due to semantic encryption and hides the location of the first element due to random rotations.
		Leakage-wise this construction is strictly better than {\BPlus} tree with \acrshort{ore} --- they both leak total order, but \cite{florian-protocol} hides distance information and smallest / largest elements.
		Specifically, for all pairs of consecutive elements $e_i$ and $e_{i+1}$ it is revealed that $e_{i+1} \ge e_i$ except for one pair of smallest and largest elements in the set.

	\subsubsection{Analysis and implementation challenges}

		Insertions are \acrshort{io}-heavy because they involve rotation of the whole data structure.
		All records will be read and written, thus the complexity is $\bigO{\nicefrac{N}{B}}$.
		Searches are faster since they involve logarithmic number of blocks.
		The first few blocks can be cached and the last substantial number of requests during the binary search will target a small number of blocks.
		The complexity is then $\bigO{\log_2 \nicefrac{N}{B}}$.

		Communication volume is small as well.
		Insertion requires $\log_2 N$ messages from each side.
		Searches require double that number because separate protocol is run for both endpoints.

		The data structure is linear in size, and the client storage is always small.
		Sizes of messages are also small as only a single ciphertext is usually transferred.

		For practitioners we have a few points.
		The construction in the original paper \cite{florian-protocol} contains a typo as $m$ and $m^\prime$ must be swapped in the insertion algorithm.
		Also, we have found some rare edge cases; when duplicate elements span over the modulo, the algorithm may not return the correct answer.
		Both inconsistencies can be fixed however.
		This protocol is not optimized for \acrshort{io} operations for insertions, and thus would be better suited for batch uploads.
