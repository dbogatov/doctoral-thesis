\subsection{Lewi-Wu \acrshort{ore}}

	\textcite{lewi-wu-ore} presented an improved version of the CLWW scheme \cite{clww-ore} which leaks strictly less.

	The novel idea was to use the ``left / right framework'' in which two ciphertexts get generated --- left and right.
	The right ciphertexts are semantically secure, so an adversary will learn nothing from them.
	Comparison is only defined between the left ciphertext of one plaintext and the right ciphertext of another plaintext.

	The approach is to split the plaintext into blocks of $d$ bits.
	The ciphertext is computed block-wise.
	For the right side, the algorithm compares the given block value to all $2^d$ possible block values; each comparison result is added (modulo 3) to a \acrshort{prf} of the previous blocks.
	All $2^d$ comparison results go into the right ciphertext.
	The left side, which is shorter, is produced in such a way as to reveal the correct comparison result.
	This way the location of the differing bit inside the block is hidden, but the location of the first differing block is revealed.

	\subsubsection{Security}
		This scheme satisfies the \acrshort{ore} security definition introduced by~\textcite{clww-ore} with the leakage $\leak(\cdot)$ of the location of the first differing \emph{block}.
		This property allows a practitioner to set performance-security tradeoff by tuning the block size.
		Left / right framework is particularly useful in a {\BPlus} tree since it is possible to store only one (semantically secure) side of a ciphertext in the structure (see \cref{section:range-snapshot:ore-to-protocol}).

	\subsubsection{Analysis and implementation challenges}

		Let $n$ be the size of input in bits (for example, 32) and $d$ be the number of bits per block (for example, 2).

		Left encryption loops $\frac{n}{d}$ times making one \acrshort{prp} call and two \acrshort{prf} calls each iteration.
		Right encryption loops $\frac{n}{d} 2^d$ times making one \acrshort{prp} call, one hash call and two \acrshort{prf} calls each iteration.
		Comparison makes $\frac{n}{d}$ calls to hash at worst and half of that number on average.
		Please note that the complexity of right encryption is exponential in the block size.
		In the \cref{table:primitive-usage-theory} the \acrshort{prp} usage is linear due to our improvement.
		The ciphertext size is no longer negligible --- $\frac{n}{d} \left(\lambda + n + 2^{d + 1} \right) + \lambda$, for $\lambda$ being \acrshort{prf} output size.

		The implementation details of this approach raise an interesting security question.
		Although the authors suggest using 3-rounds Feistel networks \cite{unbalanced-feistel} for \acrshort{prp} and use it in their implementation, it may not be secure for small input sizes.
		Feistel networks security depends on the input size \cite{feistel-security} --- exponential in the input size.
		However, the typical input for \acrshort{prp} in their scheme is 2--8 bits, thus even exponential number is small.

		We have considered multiple \acrshort{prp} implementations to use instead of the Feistel networks.
		Because the domain size is small (from $2^2$ to $2^8$ elements), we have decided to build a \acrshort{prp} by simply using the key as an index into the space of all possible permutations on the domain, where a permutation is obtained from the key via Knuth shuffle (this approach was mentioned in \cite{knuth-shuffle-security}).
		Another important aspect of the implementation is that for each block we need to compute the permutation of all the values inside the block.
		This operation applied many times can be expensive.
		To address this, we propose to generate a \acrshort{prp} table once for the whole block and then use this table when one needs to compute the location of an element of permutation.
		This can reduce the \acrshort{prp} usage (indeed, we observe a reduction from 80 to 32 calls in our case).
		We evaluate this improved approach in our experimental section.
