\section{\texorpdfstring{\acrshort{ope}}{OPE} and \texorpdfstring{\acrshort{ore}}{ORE} Schemes}

	An \acrlong{ore} scheme is a triple of polynomial\hyp{}time algorithms $\setup$, $\encrypt$ and $\compare$.
	$\setup$ generates a key of parameterized length (the $\lambda$ parameter).
	$\encrypt$ takes a numerical input (as a bit string) and produces a ciphertext.
	$\compare$ takes two ciphertexts generated by the scheme and outputs whether the first plaintext was strictly less than the second.
	Note that being able to check this condition is enough to apply all other comparison operators ($<$, $\le$, $=$, $\ge$, $>$).
	Also note that an \acrshort{ore} scheme does not include a decryption algorithm, because one can simply append a symmetric encryption of the plaintext to the produced ciphertext and use it for decryption.\footnote{
		\emph{Given the secret key}, it is possible to decrypt a ciphertext by doing binary search on the plaintext domain: encrypting known values and comparing them against the target ciphertext, until the target plaintext is found.
		However, this would require $\bigO{\log{|\domain|}}$ encryption and comparison operations.
	}
	An \acrfull{ope} scheme is a particular case of an \acrshort{ore} scheme where ciphertexts are numerical and thus $\compare$ routine is trivial (the numerical order of ciphertexts is the same as underlying plaintexts).
	\acrshort{ope} may optionally include a decryption algorithm, since appending a symmetric ciphertext is no longer possible.

	Both \acrshort{ope} and \acrshort{ore} schemes by definition allow to totally order the ciphertexts.
	This is their inherent leakage (by design) and all the \acrshort{ope} / \acrshort{ore} security definitions account for this and possibly additional leakage.

	We proceed by describing and analyzing the \acrshort{ope} / \acrshort{ore} schemes we have benchmarked.
	All plaintexts are assumed to be 32-bit signed integers, or $n$-bit inputs in complexity analysis.
	\acrshort{ope} ciphertexts are assumed to be 64-bit signed integers.

	From here, we will use the term \acrshort{ore} to refer to both \acrshort{ope} and \acrshort{ore}, unless explicitly stated otherwise.
	Each scheme has its own subsection where the first part is the construction overview followed by security discussion, and the second part is our theoretical and experimental analysis.

	\subsection{BCLO OPE}

	The OPE scheme by \textcite{bclo-ope} was the first OPE scheme that provided formal security guarantees and was used in one of the first database systems that executes queries over encrypted data (CryptDB \cite{crypt-db}).
 	The core principle of their construction is the natural connection between a random order-preserving function and the hypergeometric probability distribution.

	The encryption algorithm works by splitting the domain into two parts according to a value sampled from the hypergeometric distribution ($\hg$) routine, and splitting the range in half recursively.
	When the domain size contains a single element, the corresponding ciphertext is sampled uniformly from the current range.

	All pseudo-random decisions are made by an internal PRG ($\tapegen$ in \cite{bclo-ope}).
	This way not only the algorithm is deterministic, but also decryption is possible.
	The decryption procedure takes the same ``path'' of splitting domain and range, and when the domain size reaches one, the only value left is the original plaintext.

	\subsubsection{Security}
		This scheme is POPF-CCA secure \cite{bclo-ope}, meaning that it is as secure as the underlying ideal object --- randomly sampled order-preserving function from a certain domain to a certain range.
		For practical values of the parameters, \textcite{ope-leakage} showed that the distance between the plaintexts can be approximated to an accuracy of about the square root of the domain size.
		In other words, approximately, half of the bits (the most significant) are leaked.
		\textcite{leakage-abuse-grubs-2017} showed that this leakage allows to almost entirely decrypt the ciphertexts (given auxiliary data with a similar distribution) and encrypting strings (rather than numbers) with this scheme is especially dangerous (see \cref{section:range-snapshot:variable-inputs}).

	\subsubsection{Analysis and implementation challenges}

		Efficient sampling from the hypergeometric distribution is a challenge by itself.
		Authors suggest using a randomized yet exact (not approximate) Fortran algorithm by \textcite{hg-sampler}.
		It should be noted that the algorithm relies on infinite precision floating-point numbers, which most regular frameworks do not have.
		The security consequences of finite precision computations is actually an open question.
		The complexity of this randomized algorithm is hard to analyze; however, we empirically verified that its running time is no worse than linear in the input bit length.
		The authors also suggest a different algorithm for smaller inputs \cite{hg-sampler-small}.

		On average, encryption and decryption algorithms make $n$ calls to $\hg$, which in turn consumes entropy generated by the internal PRG\@.
		The entropy, and thus the number of calls to PRG, needed for one $\hg$ run is hard to analyze theoretically.
		However, we derived this number experimentally (see \cref{section:range-snapshot:evaluation}).

		BCLO has been implemented in numerous languages and has been deployed in a number of secure systems.
		We add {\Csharp} implementation to the list, and recommend using a library that supports infinite precision floating-point numbers when building the hypergeometric sampler.


	\subsection{CLWW \acrshort{ore}}\label{section:range-snapshot:clww}

	The \acrshort{ore} scheme by \textcite{clww-ore}, which authors call ``Practical ORE'', is the first efficient \acrshort{ore} implementation based on \acrshortpl{prf}.

	On encryption, the plaintext is split into $n$ values in the following way.
	For each bit, a value is this bit concatenated with all more significant bits.
	This value is given to a keyed \acrshort{prf} and the result is numerically added to the next less significant bit.
	This resulting list of $n$ elements is the ciphertext.

	The comparison routine traverses two lists in-order looking for the case when one value is greater than the other by exactly one, revealing location and value of the first differing bit.
	If no such index exists, the plaintexts are equal.

	\subsubsection{Security}
		A generic \acrshort{ore} security definition was introduced along with the scheme \cite{clww-ore}.
		\acrshort{ore} leakage is more clearly quantified than in \acrshort{ope}.
		The definition says that the scheme is secure with a leakage $\leak(\cdot)$ if there exists an algorithm (simulator) that has access to the leakage function and can generate output indistinguishable from the one generated by the real scheme.
		This scheme satisfies \acrshort{ore} security definition with the leakage $\leak(\cdot)$ of the location and value of the first differing bit of every pair of plaintexts.
		Note that the most significant differing bit also leaks the approximate distance between two values.

	\subsubsection{Analysis and implementation challenges}

		On encryption the algorithm makes $n$ calls to \acrshort{prf} and the comparison procedure does not use any cryptographic primitives.
		Ciphertext is a list of length $n$, where each element is an output of a \acrshort{prf} modulo 3.
		The authors claim that the ciphertext's size is $n \log_2 3$, just $1.6$ times larger than the plaintext's size.
		While this may be true for large enough $n$ if ternary encoding is used, we found that in practice the ciphertext size is still $2n$.
		$1.6 n$ for 32-bit words is $51.2$ bits, which will have to occupy one 64-bit word, or two 32-bit words, therefore resulting in $2n$ anyway.


	\subsection{Lewi-Wu ORE}

	\textcite{lewi-wu-ore} presented an improved version of the CLWW scheme~\cite{clww-ore} which leaks strictly less.

	The novel idea was to use the ``left / right framework'' in which two ciphertexts get generated --- left and right.
	The right ciphertexts are semantically secure, so an adversary will learn nothing from them.
	Comparison is only defined between the left ciphertext of one plaintext and the right ciphertext of another plaintext.

	The approach is to split the plaintext into blocks of $d$ bits.
	The ciphertext is computed block-wise.
	For the right side, the algorithm compares the given block value to all $2^d$ possible block values; each comparison result is added (modulo 3) to a PRF of the previous blocks.
	All $2^d$ comparison results go into the right ciphertext.
	The left side, which is shorter, is produced in such a way as to reveal the correct comparison result.
	This way the location of the differing bit inside the block is hidden, but the location of the first differing block is revealed.

	\subsubsection{Security}
		This scheme satisfies the ORE security definition introduced by~\textcite{clww-ore} with the leakage $\leak(\cdot)$ of the location of the first differing \emph{block}.
		This property allows a practitioner to set performance-security tradeoff by tuning the block size.
		Left / right framework is particularly useful in a {\BPlus} tree since it is possible to store only one (semantically secure) side of a ciphertext in the structure (see Section~\ref{sec:ore-to-protocol}).

	\subsubsection{Analysis and implementation challenges}

		Let $n$ be the size of input in bits (e.g.\ 32) and $d$ be the number of bits per block (e.g.\ 2).

		Left encryption loops $\frac{n}{d}$ times making one PRP call and two PRF calls each iteration.
		Right encryption loops $\frac{n}{d} 2^d$ times making one PRP call, one hash call and two PRF calls each iteration.
		Comparison makes $\frac{n}{d}$ calls to hash at worst and half of that number on average.
		Please note that the complexity of right encryption is exponential in the block size.
		In the Table~\ref{tbl:primitive-usage-theory} the PRP usage is linear due to our improvement.
		The ciphertext size is no longer negligible --- $\frac{n}{d} \left(\lambda + n + 2^{d + 1} \right) + \lambda$, for $\lambda$ being PRF output size.

		The implementation details of this approach raise an interesting security question.
		Although the authors suggest using 3-rounds Feistel networks~\cite{unbalanced-feistel} for PRP and use it in their implementation, it may not be secure for small input sizes.
		Feistel networks security depends on the input size~\cite{feistel-security} --- exponential in the input size.
		However, the typical input for PRP in their scheme is 2--8 bits, thus even exponential number is small.

		We have considered multiple PRP implementations to use instead of the Feistel networks.
		Because the domain size is small (from $2^2$ to $2^8$ elements), we have decided to build a PRP by simply using the key as an index into the space of all possible permutations on the domain, where a permutation is obtained from the key via Knuth shuffle (this approach was mentioned in~\cite{knuth-shuffle-security}).
		Another important aspect of the implementation is that for each block we need to compute the permutation of all the values inside the block.
		This operation applied many times can be expensive.
		To address this, we propose to generate a PRP table once for the whole block and then use this table when one needs to compute the location of an element of permutation.
		This can reduce the PRP usage (indeed, we observe a reduction from 80 to 32 calls in our case).
		We evaluate this improved approach in our experimental section.


	\subsection{CLOZ ORE}

	\textcite{parameter-hiding-ore} introduced a new ORE scheme that provably leaks less than any previous scheme.
	The idea is to use \textcite{clww-ore} construction (see Section~\ref{sec:clww}), but permute the list of PRF outputs.
	The original order of those outputs is not necessary, as one can simply find a pair that differs by one.
	This is not enough to reduce leakage, however, since an adversary can count how many elements two ciphertexts have in common.

	To address this problem, the authors define a new primitive they call a \emph{property-preserving hash} (PPH).
	A PPH as defined and used in~\cite{parameter-hiding-ore}, allows one to expose a property (specifically $y \overset{?}{=} x + 1$) of two (numerical) elements such that nothing else is leaked.
	In particular, the outputs are randomized, so same element hashed twice will have different hashes.
	Please refer to the original paper~\cite{parameter-hiding-ore} for formal correctness and security definitions.

	Equipped with the PPH primitive, the algorithm ``hashes'' the elements of the ciphertexts before outputting them.
	Due to security of PPH, the adversary would not be able to count how many elements two ciphertexts have in common, thus, would not be able to tell the location of differing bit.

	\subsubsection{Security}
		The strong side of the scheme is its security.
		The scheme leaks $\leak(\cdot)$ an \emph{equality pattern} of the most-significant differing bits (satisfying \textcite{clww-ore} definition).
		As defined in \cite{parameter-hiding-ore}, the intuition behind equality pattern is that for any triple of plaintexts $m_1$, $m_2$, $m_3$, it leaks whether $m_2$ differs from $m_1$ before $m_3$ does. % chktex 2
		We do not know of any attacks against this construction (partially because no implementation exists yet, see next subsection), but it is inherently vulnerable to frequency attacks that apply to all frequency-revealing ORE schemes (see Section~\ref{sec:security}).

	\subsubsection{Analysis and implementation challenges}

		On encryption, the scheme makes $n$ calls to PRF, $n$ calls to PPH \textsc{Hash} and one call to PRP\@.
		Comparison is more expensive, as the scheme makes $n^2$ calls to PPH \textsc{Test}.

		The scheme has two limitations that make it impractical.
		The first one is the square number of calls to PPH, which is around $1024$ for a single comparison.

		The second problem is the PPH itself.
		Authors suggest a construction based on bilinear maps.
		The hash of an argument is an element of a group, and the test algorithm is computing a pairing.
		This operation is very expensive --- order of magnitude more expensive than any other primitive we have implemented for other schemes.

		We have implemented this scheme in C++ using the PBC library~\cite{pbc} to empirically assess schemes's performance, and on our machine (see Section~\ref{sec:evaluation}), a single comparison takes 1.9 seconds on average.
		Although we have produced the first (correct and secure) real implementation of this scheme in C++, it is infeasible to use it in the benchmark (it will take years to complete a single run).
		Therefore, for the purposes of our benchmark, we implemented a ``fake'' version of PPH --- correct, but insecure, which does not use pairings.
		Consequently, in our analysis we did not benchmark the speed of the scheme, but measured all other data.


	\subsection{FH-OPE}

	Frequency-hiding OPE by \textcite{fh-ope} is a stateful scheme that hides the frequency of the plaintexts, so the adversary is unable to construct a frequency histogram.

	This scheme is stateful, which means that the client needs to keep a data structure and update it with every encryption and decryption.
	The data structure is a binary search tree where the node's value is the plaintext and node's position in a tree is the ciphertext.
	For example, consider the range $[1, 128]$.
	Any plaintext that happens to arrive first (for example, $6$), will be the root, and thus the ciphertext is $64$.
	Then any plaintext smaller than the root, say $3$, will become the left child of the root, and will produce the ciphertext $32$.
	To encrypt a value, the algorithm traverses the tree until it finds a spot for the new plaintext, or finds the same plaintext.
	If the same plaintext is found, the traversal pseudo-randomly passes to the left or right child, up to the leaf.
	This way, the invariant of the tree --- intervals of the same plaintexts do not overlap --- is maintained.
	The ciphertext generated from the new node's position is returned.

	Due to randomized ciphertexts, the comparison algorithm is more complicated than in the regular deterministic OPE\@.
	To properly compare ciphertexts, the algorithm needs to know the boundaries --- the minimum and maximum ciphertexts for a particular plaintext.
	The client is responsible for traversing the tree to find the plaintext for the ciphertext and then minimum and maximum ciphertext values.
	Having these values, the comparison is trivial --- equality is a check that the value is within the boundaries, and other comparison operators are similar.

	Authors have designed a number of heuristics to minimize the state size, however, these are mostly about compacting the tree and the result depends highly on the tree content.
	In our analysis, we consider the worst case performance without the use of heuristics.
	In our experimental evaluation, however, we did implement compaction.

	\subsubsection{Security}
		The security of the scheme relies on the large range size to domain size ratio.
		Authors recommend at least 6 times longer ciphertexts than the plaintexts in bit-length, which means ciphertexts should be 192-bit numbers that are not commonly supported.
		It is possible to operate over arbitrary-length numbers, but the performance overhead would be substantial.
		We did a quick micro-benchmark in {\Csharp} and the overhead of using \texttt{BigInteger} is 15--20 times for basic arithmetic operations.

		This scheme satisfies IND-FAOCPA definition (introduced along with the scheme~\cite{fh-ope}), meaning that it does not leak the equality or relative distance between the plaintexts.
		This definition has been criticized in~\cite{florian-def-critique}, who claim that the definition is imprecise and propose an enhanced definition along with a small change to construction to satisfy this new definition.
		Both schemes leak the insertion order, because it affects the tree structure.
		We do not know of any attacks against this leakage, but it does not mean they cannot exist.
		\textcite{leakage-abuse-grubs-2017} describe an attack against this scheme (binomial attack), but it applies to any perfectly secure (leaking only total order) frequency-hiding OPE\@.

	\subsubsection{Analysis and implementation challenges}

		If the binary tree grows in only one direction, at some point it will be impossible to generate another ciphertext.
		In this case, the tree has to be rebalanced.
		This procedure will invalidate all ciphertexts already generated.
		This property makes the scheme difficult to use in some protocols since they usually rely on the ciphertexts on the server being always valid.
		The authors explicitly mention that the scheme works under the assumption of uniform input.
		However, the rebalancing will be caused by insertion of just 65 consecutive input elements for 64-bit integer range.

		The scheme makes one tree traversal on encryption and decryption.
		Comparison is trickier as it requires one traversal to get the plaintext, and two traversals for minimum and maximum ciphertexts.
		We understand that it is possible to get these values in fewer than three traversals, but we did not optimize the scheme for the analysis and evaluation.

		For practitioners we note that the stateful nature of the scheme implies that the client storage is no longer negligible as the state grows proportionally to the number of encryptions.
		We also note that implementing compaction extensions will affect code complexity and performance.
		Finally, we stress again that some non-uniform inputs can break the scheme by causing all ciphertexts to be invalid.
		It is up to the users of the scheme to ensure uniformity of the input, which poses serious restrictions on the usage of the scheme.


	% cSpell:ignore Lewi BCLO CLWW captionsetup

\begin{sidewaystable}
	\renewcommand{\arraystretch}{1.5}
	\centering
	\captionsetup{width=\textwidth}
	\caption[Primitive usage by \acrshort{ope} / \acrshort{ore} schemes]{
		\cite[Tables 1 and 4]{ore-benchmark-17}.
		Primitive usage by \acrshort{ope} / \acrshort{ore} schemes.
		Ordered by security rank --- most secure below.
		$n$ is the input length in bits, $d$ is a block size for Lewi-Wu \cite{lewi-wu-ore} scheme, $\lambda$ is a \acrshort{prf} output size, $N$ is a total data size, \textbf{\acrshort{hg}} is a hyper-geometric distribution sampler, \textbf{\acrshort{pph}} is a property-preserving hash with $h$-bit outputs built with bilinear maps and \textbf{bolded} are weak points of the schemes.
		Values in parentheses are simulation-derived. $N = 10^3$, $n = 32$, $d = 2$, $\lambda = 128$ and $h = 128$ in this simulation.
	}\label{tbl:primitive-usage-theory}
	\begin{tabular*}{\linewidth}{ !{\extracolsep\fill} l c c c c c } % chktex 26

		\toprule

		\multirow{2}{*}{Scheme}						& \multicolumn{2}{c}{Primitive usage (number of invocations)}																				& \multirow{2}{*}{\makecell{Ciphertext size, \\ or state size (bits)}}					&  \multirow{2}{*}{\makecell{Leakage \\ (in addition to inherent total order)}}				\\ \cline{2-3}
		\rule{0pt}{10pt}							& Encryption																& Comparison													& 																						& 																							\\

		\toprule

		\cite{bclo-ope}								& $\bm{\approx n}$ \textbf{(41) \acrshort{hg}}								& none															& $2n$ (64)																				& \textbf{$\approx$ Top half of the bits}													\\

		\midrule

		\cite{clww-ore}								& $n$ (32) \acrshort{prf} 													& none															& $2n$ (64)																				& \textbf{Most-significant differing bit}													\\

		\midrule

		\multirow{3}{*}{\cite{lewi-wu-ore}}			& \boldmath{} $\nicefrac{2n}{d}$ \unboldmath{} \textbf{(32) \acrshort{prp}}	& \multirow{3}{*}{$\frac{n}{2d}$ (9) Hash}						& 																						& \multirow{3}{*}{Most-significant differing block}											\\
													& $2 \frac{n}{d} \left( 2^d + 1 \right)$ (160) \acrshort{prf}				&																& $\frac{n}{d} \left(\lambda + n + 2^{d + 1} \right) + \lambda$							&																							\\
													& $\frac{n}{d} 2^d$ (64) Hash												&																& (2816)																				&																							\\


		\midrule

		\multirow{3}{*}{\cite{parameter-hiding-ore}}			& $n$ (32) \acrshort{prf}													& \multirow{3}{*}{$\bm{n^2}$ \textbf{(1046) \acrshort{pph}}}	& \multirow{3}{*}{$n \cdot h$ (4096)}													& \multirow{3}{*}{\makecell{Equality pattern \\ of the most-significant \\ differing bit}}	\\
													& $n$ (32) \acrshort{pph}													&																&																						& 																							\\
													& 1 \acrshort{prp}															&																&																						& 																							\\

		\midrule

		\cite{fh-ope}								& 1 Traversal																& 3 Traversals													& $\bm{3 \cdot n \cdot N}$ \textbf{(86842)}												& Insertion order																			\\

		\bottomrule

	\end{tabular*}
\end{sidewaystable}


\section{Secure Range Query Protocols}

	We proceed by describing and analyzing the range query protocols we have chosen.
	For the purpose of this paper, a secure range-query protocol is defined as a client-server communication involving construction and search stages.
	Communication occurs between a client, who owns some sensitive data, and an honest server, who securely stores it.
	In construction stage, a client sends the server the encrypted datapoints (index-value tuples) and the server stores them in some internal data structure.
	In search stage, a client asks the server for a range (usually specifying it with encrypted endpoints) and the server returns a set of encrypted records matching the query.
	Note that the server may interact with the client during both stages (e.g.\ ask the client to sort a small list of ciphertexts).
	Also note that we do not allow batch insertions as it would limit the use cases (e.g.\ client may require interactive one-by-one insertions).

	The first protocol is a family of constructions where a data structure ({\BPlus} tree in this case) uses \acrshort{ore} schemes internally.
	Then, we present alternative solutions with varying performance and security profiles, not relying on \acrshort{ore}.
	Finally, we introduce two baseline solutions we will use in the benchmark --- one that achieves the best performance and the other that achieves the maximal security.

	\subsection{Range query protocol from \texorpdfstring{\acrshort{ore}}{ORE}}\label{section:range-snapshot:ore-to-protocol}

	So far we have analyzed \acrshort{ope} and \acrshort{ore} schemes without much context.
	One of the best uses of an \acrshort{ore} is within a secure protocol.
	In this section we provide a construction of a search protocol built with a {\BPlus} tree working on top of an \acrshort{ore} scheme and analyze its security and performance.

	The general idea is to consider some data structure that is optimized for range queries, and to modify it to change all comparison operators to \acrshort{ore} scheme's $\compare$ calls.
	This way the data structure can operate only on ciphertexts.
	Performance overhead would be that of using the \acrshort{ore} scheme's $\compare$ routine instead of a plain comparison.
	Space overhead would be that of storing ciphertexts instead of plaintexts.

	In this paper, we have implemented a typical {\BPlus} tree \cite{b-tree} (with a proper deletion algorithm \cite{b-plus-tree-deletion}) as a data structure.

	For protocols, we also analyze the \acrshort{io} performance and the communication cost.
	In particular, we are interested in the expected number of \acrshort{io} requests the server would have made to the secondary storage, and the number and size of messages parties would have exchanged.

	The relative performance of the {\BPlus} tree depends only on the page capacity (the longer the ciphertexts, the smaller the branching factor). 	Therefore, the query complexity is $\bigO{\log_B \left( \nicefrac{N}{B} \right) + \nicefrac{r}{B}}$, where $B$ is the number of records (ciphertexts) in a block, $N$ is the number of records (ciphertexts) in the tree and $r$ is the number of records (ciphertexts) in the result (none for insertions).

	Communication amount of the protocol is relatively small as its insertions and queries require at most one round trip.

	\subsubsection{Security}
		The leakage of this protocol consists of leakage of the underlying \acrshort{ore} scheme plus whatever information about insertion order is available in the {\BPlus} tree.
		Please note that Lewi-Wu \cite{lewi-wu-ore} \acrshort{ore} is particularly well-suited in this construction with its left / right framework, because only the semantically secure side of the ciphertext is stored in the structure.
		In this case, the \acrshort{ore} leakage becomes only the total order and the security of the protocol is comparable with other non-\acrshort{ore} constructions.


	\subsection{Kerschbaum-Tueno}

	\textcite{florian-protocol} proposed a new data structure, which satisfies their own definitions of security (IND-CPA-DS) and efficiency (search operation has poly\hyp{}logarithmic running time and linear space complexity).

	In short, the idea is to maintain a (circular) array of symmetrically encrypted ciphertexts in order.
	On insertion, the array is rotated around a uniformly sampled offset to hide the location of the smallest element.
	Client interactively performs a binary search requesting an element, decrypting it and deciding which way to go.

	\subsubsection{Security}
		Authors prove that this construction is IND-CPA-DS secure (defined in the same paper \cite{florian-protocol}).
		The definition assumes an array data structure and therefore serves specifically this construction (as opposed to being generic).
		It provably hides the frequency due to semantic encryption and hides the location of the first element due to random rotations.
		Leakage-wise this construction is strictly better than {\BPlus} tree with ORE --- they both leak total order, but \cite{florian-protocol} hides distance information and smallest / largest elements.
		Specifically, for all pairs of consecutive elements $e_i$ and $e_{i+1}$ it is revealed that $e_{i+1} \ge e_i$ except for one pair of smallest and largest elements in the set.

	\subsubsection{Analysis and implementation challenges}

		Insertions are {\IO}-heavy because they involve rotation of the whole data structure.
		All records will be read and written, thus the complexity is $\bigO{\nicefrac{N}{B}}$.
		Searches are faster since they involve logarithmic number of blocks.
		The first few blocks can be cached and the last substantial number of requests during the binary search will target a small number of blocks.
		The complexity is then $\bigO{\log_2 \nicefrac{N}{B}}$.

		Communication volume is small as well.
		Insertion requires $\log_2 N$ messages from each side.
		Searches require double that number because separate protocol is run for both endpoints.

		The data structure is linear in size, and the client storage is always small.
		Sizes of messages are also small as only a single ciphertext is usually transferred.

		For practitioners we have a few points.
		The construction in the original paper \cite{florian-protocol} contains a typo as $m$ and $m^\prime$ must be swapped in the insertion algorithm.
		Also, we have found some rare edge cases; when duplicate elements span over the modulo, the algorithm may not return the correct answer.
		Both inconsistencies can be fixed however.
		This protocol is not optimized for {\IO} operations for insertions, and thus would be better suited for batch uploads.


	\subsection{POPE}

	\textcite{pope} presented a protocol, direct improvement over mOPE \cite{ope-ideal-security-protocol}, which is especially suitable for large number of insertions and small number of queries.
	The construction is heavily based on buffer trees \cite{buffer-tree} to support fast insertion and lazy sorting.

	The idea is to maintain a POPE tree on the server and have the client manipulate that tree.
	POPE tree is similar to B-tree, in that the nodes have multiple children and nodes are sorted on each level.
	Each node has an ordered list of \emph{labels} of size $L$ and an unbounded unsorted set of encrypted data called buffer.
	Parameter $L$ controls the list size, the leaf's buffer size, and the size of client's working set.
	The insertion procedure simply adds an encrypted piece of data to the root's buffer, thus we do not concentrate on insertion analysis in this section.

	The query procedure is more complex.
	To answer a query, the server interacts with the client to split the tree according to the query endpoints.
	On a high level, for each endpoint the buffers are cleared (content pushed down to leaves), and nodes in the paths are split.
	After that, answering a query means replying with all ciphertexts in all buffers between the two endpoint leaves.

	The authors provide cost analysis of their construction.
	Search operations are expected to require $\bigO{\log_L n}$ rounds.
	It must be noted that the first queries will require many more rounds, since large buffers must be sorted.

	\subsubsection{Security}
		This construction satisfies the security definition of frequency\hyp{}hiding partial order-preserving (FH-POP) protocol (introduced in the paper \cite{pope}).
		According to \cite[Theorem~3]{pope}, after $n$ insertions and $m$ queries with local storage of size $L$, where $m L \in o(n)$, the POPE scheme is frequency-hiding partial order-preserving with $\bigOmega{ \frac{n^2}{mL \log_L n} - n }$ incomparable pairs of elements. % chktex 2
		Simply put, the construction leaks pairwise order of a \emph{bounded} number of elements.
		Aside from this, the construction provably hides the frequency (i.e.\ equality) of the elements.

	\subsubsection{Analysis and implementation challenges}

		In our analysis we count each request-response communication as a round.
		This is different from \cite{pope} where they use \emph{streaming} a number of elements as a single round. % chktex 2
		The rationale for our approach is that if we allow persistent channels additionally to messages, then any protocol can open a channel for each operation.
		Thus, we do not allow channels for all protocols in our analysis.

		Also, as noted by the authors, if $L = n^{\epsilon}$ for $0 < \epsilon < 1$, then the amortized costs become $\bigO{1}$.
		While this is true, in our analysis the choice of $L$ depends on the storage volume block size for {\IO} optimizations, instead of the client's volatile storage capacity.
		Thus, the costs remain logarithmic.

		Search bandwidth depends heavily on the current state of the tree.
		When the tree is completely unsorted (the first query), all elements of the tree will be transferred to split the large root, then possibly internal node will have to be split requiring sending of $\frac{N}{L}$ elements, and so on, thus $\bigO{N + r}$.
		When the tree is completely sorted (after a large number of uniform queries), the bandwidth will be similar to that of a standard {\BPlus} tree --- $\bigO{L \log_L N + r}$.
		The average case is hard to compute; however, authors prove an upper bound on bandwidth after $n$ insertions and $m$ queries --- $\bigO{m L \log_L n + n \log_L m + n \log_L (\lg n) }$.

		POPE tree is not optimized for {\IO} the way B-tree is.
		Search complexity is hard to analyze as is bandwidth complexity.
		In the worst-case (first query), all blocks need to be accessed $\bigO{\frac{N}{B} + \frac{r}{B}}$.
		In the best-case all nodes occupy exactly one block and {\IO} complexity is the same as with {\BPlus} tree $\bigO{\log_L \frac{N}{B} + \frac{r}{B}}$.
		The average case is in between and matters get worse as the node is not guaranteed to occupy a single block due to the buffers of arbitrary size.

		Client's persistent storage is negligibly small --- it stores the encryption key.
		Volatile storage is bounded by $L$.

		For practitioners we present a number of things to consider.
		Buffer within one node is unsorted, so in the worst-case, $L$-sized chunks remain unordered.
		Due to this property, the query result may contain up to $2 (L - 1)$ extra entries, which the client will have to discard from the response.

		The first query after a large number of insertions will result in client sorting the whole $N$ elements, and thus, POPE has different performance for cold and warm start.
		Also, even to navigate an already structured tree, the server has to send to the client the whole $L$ elements and ask where to go on all levels.

		Furthermore, \cite{pope} does not stress the fact that after alternating insertions and queries, it may happen that some intermediate buffers are not empty, thus returning buffers between endpoints must include intermediate buffers as well. % chktex 2
		The consequence is that the whole subtree is traversed between paths to endpoints, unlike the {\BPlus} tree case where only leaves are involved.

		Finally, POPE tree is not optimized for {\IO} operations.
		Even if $L$ is chosen so that the node fits in the block, only leaves and only after some number of searches will optimally fit in blocks.
		Arbitrary sized buffers of intermediate nodes and the lack of underflow requirement do not allow for {\IO} optimization.


	\subsection{Logarithmic-BRC}

	\textcite{practical-range-search} introduced a novel protocol called ``Logarithmic-BRC'' whose \acrshort{io} complexity depends only on the result size, regardless of the database size.
	The core primitive for their construction is a \acrfull{sse} scheme.
	An \acrshort{sse} scheme is a server-client protocol in which the server stores a specially encrypted keywords-to-documents map, and a client can query documents with keywords while the server
	learns neither keywords nor the documents.
	Note that the map stores short document identifiers instead of the actual documents, and we will use the term ``documents'' to mean ``document identifiers'' or ``record IDs'' in this section.

	The construction treats record values as documents and index ranges as keywords so that records can be retrieved by the ranges that include them.
	Specifically, a client builds a virtual binary tree over the domain of indices and assigns each record a set of keywords, which is the path from that record to the root.
	This way, the root keyword is associated with all documents and the leaf keyword is associated with only one record.

	Upon query, a client computes a cover --- a set of nodes whose sub-trees cover the requested range.
	A client sends these keywords to the \acrshort{sse} server, which returns encrypted documents --- result values.
	Of the several covering techniques suggested in the protocol \cite{practical-range-search} we have chosen the \acrfull{brc}, because it results in fewest nodes and does not return false-positives.
	\textcite{brc} have proven that the worst-case number of nodes for domain of size $N$ is $\bigO{\log N}$ and presented an efficient \acrshort{brc} algorithm.

	\subsubsection{Security}

		In a snapshot setting, this construction's security is that of the \acrshort{sse}.
		We have used \cite{cjjkrs-13} and \cite{cjjjkrs-14} \acrshort{sse} schemes; their leakage in a snapshot setting is the database size and at most some initialization parameters.
		Thus, the security of these schemes is high enough to call them \emph{fully hiding} in our setting.
		Additional access pattern leakage comes up during queries; exact implications of this leakage remain an open research problem but it is known that it can be harmful \cite{generic-attacks-kellaris}.

	\subsubsection{Analysis and implementation challenges}

		Communication involves a client sending at worst $\log_2 N$ keywords and server responding with the exact result.

		For each keyword in the query set, server will query the \acrshort{sse} scheme, which will return $r$ documents.
		Therefore, server's \acrshort{io} complexity is that of \acrshort{sse}.

		\textcite{practical-range-search} have used \cite{cjjkrs-13} \acrshort{sse} scheme in their implementation, but we have found it slow it terms of \acrshort{io}.
		Instead, we have implemented an improved scheme \cite{cjjjkrs-14}, which directly addresses \acrshort{io} optimization.

		Both \acrshort{sse} schemes' \acrshort{io} complexity is linear with the result size $r$.
		\cite{cjjjkrs-14} scheme makes at most one \acrshort{io} per result document in the worst-case and there are extensions to significantly improve \acrshort{io} complexity. % chktex 2
		We have implemented the \texttt{pack} extension, which packs documents in blocks to fit the \acrshort{io} pages.
		We note that this extension can dramatically reduce the \acrshortpl{io} (see \cref{section:range-snapshot:results-protocols} and \cref{figure:protocols-query-sizes}).

		Logarithmic-BRC is very scalable as its performance does not depend on total data size and only degrades with the result size.
		Storage overhead, however, is significant.
		Each record is associated with the whole path in the binary tree --- $\log_2 N$ nodes (keywords).
		The storage complexity is therefore $\bigO{N \log N}$, and the overhead is then a factor of $\log N$.

		Updates, while addressed in the original protocol, are not very practical in this construction.
		Authors suggest using bulk-loading for updates, maintaining merge trees, and requiring the client to do a merge once in a while.
		The \acrshort{io} complexity of such approach is unclear.
		In our implementation we perform the construction stage only in batch mode, and thus do not include it in the analysis.
		We also emphasize that the update routine was not implemented for evaluation in the original paper.


	\subsection{The two extremes}

	To put the aforementioned protocols in a context we introduce the baselines --- an efficient and insecure construction we will refer to as \emph{no encryption} and maximal security protocol we refer to as \emph{\acrshort{oram}}.

	\subsubsection{No encryption}

		This protocol is a regular {\BPlus} tree \cite{b-tree} without any \acrshort{ore} in it.
		It is the construction one can expect to see in almost any general-purpose database.

		In terms of security it provides no guarantees --- all data is in the clear.
		In terms of efficiency it is optimal.
		{\BPlus} tree data structure is optimal in \acrshort{io} usage, indices inside nodes are smallest possible (integers) and there is no overhead in comparing elements inside the nodes as opposed to working with \acrshort{ore} ciphertexts.

	\subsubsection{\texorpdfstring{\acrshort{oram}}{ORAM}}\label{section:range-snapshot:oram}

		\acrfull{oram} is a construction that additionally to semantic security of a snapshot setting (see \cref{section:range-snapshot:security}) provably hides the access pattern --- a sequence of reads and writes to particular memory locations.
		With \acrshort{oram} an adversary would not be able to recognize a series of accesses to the same location and will not differentiate reads versus writes.
		\acrshort{oram} was introduced by~\textcite{oram-original} who also proved its lower bound (strengthened in \cite{oram-tighter-lower-bound}) --- logarithmic overhead per request.
		A number of efficient \acrshort{oram} constructions were designed (see \cite{oram-survey-feifei} for a good survey) and we use the state-of-the-art construction, PathORAM \cite{path-oram}.

		A generic \acrshort{oram} server responds to read and write requests for a particular address.
		In our baseline protocol we store {\BPlus} tree nodes in \acrshort{oram}.
		A client works with the tree as it normally would except each time it needs to access a node, it communicates with \acrshort{oram}.

		In terms of security this protocol is fully hiding in the snapshot model and provably hides the access pattern.
		We note that one can improve security even further by adding noise to the result obscuring communication volume.
		We also note that a practitioner can use a similar protocol with \acrshort{oram} replaced with a trivial data store and have the tree nodes encrypted.
		It would be fully hiding in a snapshot setting, but we prefer the baseline that covers more than only the snapshot model.

		In terms of performance this construction incurs some noticeable overhead.
		Regardless of specific \acrshort{oram} being used, each access incurs at least logarithmic overhead according to lower bounds \cite{oram-original}.
		Combined with logarithmic complexity of the {\BPlus} tree itself, the complexity, both \acrshort{io} and communication, is $\bigO{\log^2 N}$.
		We found that PathORAM has good \acrshort{io} performance, as its internal tree structure translates into good cache affinity.
		Unlike in other protocols in our benchmark, \acrshort{oram} client does most of the computational work.
		While the server only makes \acrshort{io} requests, the client handles encryption, shuffling, and request logic.

		We present this protocol as a baseline solution in terms of security over efficiency.
		We have not implemented stand-alone PathORAM, but rather a simulator which correctly reports \acrshort{io}, communication and primitive usage.
		Surprisingly, we found that \acrshort{oram} protocol's overhead, although higher than in \acrshort{ore}-based protocols, is in-line with the most secure protocols in our benchmark.


	% cSpell:ignore Kerschbaum captionsetup

\begin{sidewaystable}
	\renewcommand{\arraystretch}{1.5}
	\centering
	\captionsetup{width=\textwidth}
	\caption[Performance of the range query protocols]{
		\cite[Tables 2 and 3]{ore-benchmark-17}.
		Performance of the range query protocols.
		Ordered by security rank --- most secure below.
		$N$ is a total data size, $B$ is an \acrshort{io} page size, $L$ is a POPE tree branching factor, $r$ is the result size in records and \textbf{bolded} are weak points of the protocols.
		The cell content is structured as follows: top value is the analytical result in $\mathcal{O}$ notation, bottom value is the number of requests for \acrshort{io} requests or number of messages and their total size for communication.
		In these experiments, $N = 247K$, $B = \SI{4}{\kilo\byte}$, $r \approx 247K \cdot \SI{0.5}{\percent} = 1235$, and $L = 60$.
	}\label{table:protocols}
	\begin{tabular*}{\linewidth}{ !{\extracolsep\fill} c c c c c c } % chktex 26

		\toprule

		\multirow{2}{*}{Protocol}			& \multicolumn{2}{c}{\acrshort{io} requests}																																			& \multirow{2}{*}{Leakage}						& \multicolumn{2}{c}{Communication}																																										\\ \cline{2-3} \cline{5-6}
		\rule{0pt}{10pt}					& Construction														& Query																												&												& Construction																						& Query 																							\\

		\toprule

		\makecell{{\BPlus} tree \\ with \acrshort{ore}}	& \makecell{$\log_B \frac{N}{B}$ \\ 3 requests}						& \makecell{$\log_B \frac{N}{B} + \frac{r}{B}$ \\ 44 requests}														& \textbf{Same as \acrshort{ore}}				& \makecell{$1$ \\ 2 / \SI{177}{\byte}}																& \makecell{$1$ \\ 2 / \SI{342}{\byte}}																\\

		\midrule

		\cite{florian-protocol}				& \makecell{$\bm{\frac{N}{B}}$ \\ \textbf{494 requests}}			& \makecell{$\log_2 \frac{N}{B} + \frac{r}{B}$ \\ 17 requests}														& \textbf{Total order}							& \makecell{$\log_2 N$ \\ 40 / \SI{671}{\byte}}														& \makecell{$\log_2 N$ \\ 86 / \SI{1453}{\byte}}													\\

		\midrule

		\makecell{\cite{pope} \\ warm}					& \multirow{2}{*}{\makecell{$1$ \\ 1 request}}						& \makecell{$\log_L \frac{N}{B} + \frac{r}{B}$ \\ 300 requests}														& \textbf{Partial order}						& \multirow{2}{*}{\makecell{$1$ \\ 2 / \SI{32}{\byte}}}												& \makecell{$\log_L N$ \\ 914 / \SI{347}{\kilo\byte}}												\\ \cline{1-1} \cline{3-3} \cline{6-6}

		\makecell{\cite{pope} \\ cold}					& 																	& \makecell{$\bm{{\nicefrac{N}{B}}}$ \\ \textbf{2175 requests}}														& Fully hiding									& 																									& \makecell{$\bm{N}$ \\ \textbf{498K / \SI[detect-all=true]{9}{\mega\byte}}}						\\

		\midrule

		\cite{practical-range-search}		& \textbf{---}														& \makecell{$\bm{r}$ \\ \textbf{40 requests}}																		& Same as \acrshort{sse}						& \textbf{---}																						& \makecell{$\log_2 N$ \\ 2 / \SI{391}{\byte}}														\\

		\midrule

		\acrshort{oram}						& \makecell{$\bm{{ \log^2 \frac{N}{B} }}$ \\ \textbf{31 requests}}	& \makecell{$\bm{{ \log_2 \frac{N}{B} \left( \log_B \frac{N}{B} + \frac{r}{B} \right) }}$ \\ \textbf{185 requests}}	& \makecell{Fully hiding \\ (access pattern)}	& \makecell{$\bm{{ \log^2 \frac{N}{B} }}$ \\ \textbf{143 / \SI[detect-all=true]{18}{\kilo\byte}}}	& \makecell{$\bm{{ \log^2 \frac{N}{B} }}$ \\ \textbf{490 / \SI[detect-all=true]{63}{\kilo\byte} }}	\\

		\bottomrule

	\end{tabular*}
\end{sidewaystable}

