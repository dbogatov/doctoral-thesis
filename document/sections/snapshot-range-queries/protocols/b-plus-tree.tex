\subsection{Range query protocol from ORE}\label{sec:ore-to-protocol}

	So far we have analyzed OPE and ORE schemes without much context.
	One of the best uses of an ORE is within a secure protocol.
	In this section we provide a construction of a search protocol built with a {\BPlus} tree working on top of an ORE scheme and analyze its security and performance.

	The general idea is to consider some data structure that is optimized for range queries, and to modify it to change all comparison operators to ORE scheme's $\compare$ calls.
	This way the data structure can operate only on ciphertexts.
	Performance overhead would be that of using the ORE scheme's $\compare$ routine instead of a plain comparison.
	Space overhead would be that of storing ciphertexts instead of plaintexts.

	In this paper, we have implemented a typical {\BPlus} tree \cite{b-tree} (with a proper deletion algorithm \cite{b-plus-tree-deletion}) as a data structure.

	For protocols, we also analyze the {\IO} performance and the communication cost.
	In particular, we are interested in the expected number of {\IO} requests the server would have made to the secondary storage, and the number and size of messages parties would have exchanged.

	The relative performance of the {\BPlus} tree depends only on the page capacity (the longer the ciphertexts, the smaller the branching factor). 	Therefore, the query complexity is $\bigO{\log_B \left( \nicefrac{N}{B} \right) + \nicefrac{r}{B}}$, where $B$ is the number of records (ciphertexts) in a block, $N$ is the number of records (ciphertexts) in the tree and $r$ is the number of records (ciphertexts) in the result (none for insertions).

	Communication amount of the protocol is relatively small as its insertions and queries require at most one round trip.

	\subsubsection{Security}
		The leakage of this protocol consists of leakage of the underlying ORE scheme plus whatever information about insertion order is available in the {\BPlus} tree.
		Please note that Lewi-Wu \cite{lewi-wu-ore} ORE is particularly well-suited in this construction with its left / right framework, because only the semantically secure side of the ciphertext is stored in the structure.
		In this case, the ORE leakage becomes only the total order and the security of the protocol is comparable with other non-ORE constructions.
