\subsection{CLOZ ORE}

	\textcite{parameter-hiding-ore} introduced a new ORE scheme that provably leaks less than any previous scheme.
	The idea is to use \textcite{clww-ore} construction (see Section~\ref{sec:clww}), but permute the list of PRF outputs.
	The original order of those outputs is not necessary, as one can simply find a pair that differs by one.
	This is not enough to reduce leakage, however, since an adversary can count how many elements two ciphertexts have in common.

	To address this problem, the authors define a new primitive they call a \emph{property-preserving hash} (PPH).
	A PPH as defined and used in~\cite{parameter-hiding-ore}, allows one to expose a property (specifically $y \overset{?}{=} x + 1$) of two (numerical) elements such that nothing else is leaked.
	In particular, the outputs are randomized, so same element hashed twice will have different hashes.
	Please refer to the original paper~\cite{parameter-hiding-ore} for formal correctness and security definitions.

	Equipped with the PPH primitive, the algorithm ``hashes'' the elements of the ciphertexts before outputting them.
	Due to security of PPH, the adversary would not be able to count how many elements two ciphertexts have in common, thus, would not be able to tell the location of differing bit.

	\subsubsection{Security}
		The strong side of the scheme is its security.
		The scheme leaks $\leak(\cdot)$ an \emph{equality pattern} of the most-significant differing bits (satisfying \textcite{clww-ore} definition).
		As defined in \cite{parameter-hiding-ore}, the intuition behind equality pattern is that for any triple of plaintexts $m_1$, $m_2$, $m_3$, it leaks whether $m_2$ differs from $m_1$ before $m_3$ does. % chktex 2
		We do not know of any attacks against this construction (partially because no implementation exists yet, see next subsection), but it is inherently vulnerable to frequency attacks that apply to all frequency-revealing ORE schemes (see Section~\ref{sec:security}).

	\subsubsection{Analysis and implementation challenges}

		On encryption, the scheme makes $n$ calls to PRF, $n$ calls to PPH \textsc{Hash} and one call to PRP\@.
		Comparison is more expensive, as the scheme makes $n^2$ calls to PPH \textsc{Test}.

		The scheme has two limitations that make it impractical.
		The first one is the square number of calls to PPH, which is around $1024$ for a single comparison.

		The second problem is the PPH itself.
		Authors suggest a construction based on bilinear maps.
		The hash of an argument is an element of a group, and the test algorithm is computing a pairing.
		This operation is very expensive --- order of magnitude more expensive than any other primitive we have implemented for other schemes.

		We have implemented this scheme in C++ using the PBC library~\cite{pbc} to empirically assess schemes's performance, and on our machine (see Section~\ref{sec:evaluation}), a single comparison takes 1.9 seconds on average.
		Although we have produced the first (correct and secure) real implementation of this scheme in C++, it is infeasible to use it in the benchmark (it will take years to complete a single run).
		Therefore, for the purposes of our benchmark, we implemented a ``fake'' version of PPH --- correct, but insecure, which does not use pairings.
		Consequently, in our analysis we did not benchmark the speed of the scheme, but measured all other data.
