\subsection{The two extremes}

	To put the aforementioned protocols in a context we introduce the baselines --- an efficient and insecure construction we will refer to as \emph{no encryption} and maximal security protocol we refer to as \emph{ORAM}.

	\subsubsection{No encryption}

		This protocol is a regular {\BPlus} tree \cite{b-tree} without any \acrshort{ore} in it.
		It is the construction one can expect to see in almost any general-purpose database.

		In terms of security it provides no guarantees --- all data is in the clear.
		In terms of efficiency it is optimal.
		{\BPlus} tree data structure is optimal in \acrshort{io} usage, indices inside nodes are smallest possible (integers) and there is no overhead in comparing elements inside the nodes as opposed to working with \acrshort{ore} ciphertexts.

	\subsubsection{\acrshort{oram}}\label{section:range-snapshot:oram}

		\acrfull{oram} is a construction that additionally to semantic security of a snapshot setting (see \cref{section:range-snapshot:security}) provably hides the access pattern --- a sequence of reads and writes to particular memory locations.
		With \acrshort{oram} an adversary would not be able to recognize a series of accesses to the same location and will not differentiate reads versus writes.
		\acrshort{oram} was introduced by~\textcite{oram-original} who also proved its lower bound (strengthened in \cite{oram-tighter-lower-bound}) --- logarithmic overhead per request.
		A number of efficient \acrshort{oram} constructions were designed (see \cite{oram-survey-feifei} for a good survey) and we use the state-of-the-art construction, PathORAM \cite{path-oram}.

		A generic \acrshort{oram} server responds to read and write requests for a particular address.
		In our baseline protocol we store {\BPlus} tree nodes in \acrshort{oram}.
		A client works with the tree as it normally would except each time it needs to access a node, it communicates with \acrshort{oram}.

		In terms of security this protocol is fully hiding in the snapshot model and provably hides the access pattern.
		We note that one can improve security even further by adding noise to the result obscuring communication volume.
		We also note that a practitioner can use a similar protocol with \acrshort{oram} replaced with a trivial data store and have the tree nodes encrypted.
		It would be fully hiding in a snapshot setting, but we prefer the baseline that covers more than only the snapshot model.

		In terms of performance this construction incurs some noticeable overhead.
		Regardless of specific \acrshort{oram} being used, each access incurs at least logarithmic overhead according to lower bounds \cite{oram-original}.
		Combined with logarithmic complexity of the {\BPlus} tree itself, the complexity, both \acrshort{io} and communication, is $\bigO{\log^2 N}$.
		We found that PathORAM has good \acrshort{io} performance, as its internal tree structure translates into good cache affinity.
		Unlike in other protocols in our benchmark, \acrshort{oram} client does most of the computational work.
		While the server only makes \acrshort{io} requests, the client handles encryption, shuffling, and request logic.

		We present this protocol as a baseline solution in terms of security over efficiency.
		We have not implemented stand-alone PathORAM, but rather a simulator which correctly reports \acrshort{io}, communication and primitive usage.
		Surprisingly, we found that \acrshort{oram} protocol's overhead, although higher than in \acrshort{ore}-based protocols, is in-line with the most secure protocols in our benchmark.
