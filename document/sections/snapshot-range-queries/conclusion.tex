\section{Remarks and conclusion}\label{section:range-snapshot:conclusion}

	Having done theoretical and practical evaluations of the protocols, we have found that primitive usage is a much better performance measure than the plain time measurements.
	When it comes to practical use, the observed time of a query execution is a mix of a number of factors and {\IO} requests can slow the system down dramatically.

	ORE-based {\BPlus} tree protocol is provably {\IO} optimal and can potentially be extended by using another data structure with ORE\@.
	Its security/performance trade off is tunable by choosing and parametrizing the underlying ORE scheme.
	Each scheme we considered has its own unique advantages and drawbacks.
	BCLO \cite{bclo-ope} is the least secure scheme in the benchmark, but is stateless and produces numerical ciphertexts, so it may be used in the databases without any modifications.
	Frequency-hiding OPE \cite{fh-ope} also has this property, hides the frequency of the ciphertexts, but is stateful and requires uniform input.
	Lewi-Wu \cite{lewi-wu-ore} is easily customizable in terms of tuning performance to security ratio, and it offers the security benefits of left / right framework --- particuarly useful for {\BPlus} tree.
	CLWW \cite{clww-ore} provides weaker security guarantees but is the fastest scheme in the benchmark.

	Kerschbaum protocol \cite{florian-protocol} offers semantically secure ciphertexts, hiding the location of the smallest and largest of them, and has a simple implementation.
	The protocol is well-suited for bulk insertions and scales well.

	\balance%

	POPE \cite{pope} offers a ``deferred'' {\BPlus} tree implementation.
	By deferring the sorting of its ciphertexts, POPE remains more secure for the small number of queries.
	POPE has the fastest insertion routine and does not reveal the order of most of its ciphertexts.
	It will be more performant for the systems where there are a lot more insertions than queries.
	We would also recommend to ``warm up'' the structure to avoid a substantial delay upon the first query.

	Logarithmic\hyp{}BRC is a perfect choice for huge datasets where query result size is limited.
	It is the only protocol with substantial space overhead, but it offers scalability and perfect (in a snapshot setting) security,
	and a carefully chosen and configured SSE scheme ensures that {\IO} grows slowly as a function of result size.

	ORAM has shown the most interesting result.
	Its performance is not only adequate, but also in-line with the other even less secure protocols.
	With this empirical result, we expect more interest in ORAM research, possibly discovering tighter bounds, faster constructions and efficient ways to use the schemes.
	The performance of ORAM gives an upper bound on the acceptable performance level of less secure (access pattern revealing) protocols, as practitioners will choose ORAM over both less secure and less performant solutions.

	We found our framework to be a powerful tool for analyzing the protocols, and we hope developers of new protocols will contribute implementations and evaluate them.

	An important future work is to understand better the meaning of the different leakage profiles and their implications.
	Furthermore, another direction is to try to improve the performance of the most secure schemes (e.g. \cite{parameter-hiding-ore}). % chktex 2
