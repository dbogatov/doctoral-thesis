\section{Security Perspective}\label{sec:security}

	Each scheme and protocol we analyze has its own security definition, which captures different leakage levels.
	We attempt to unify these definitions and analyze them under a common framework.
	We also attempt to assess relative security of these definitions and analyze their leakages.

	In this work we mostly consider the snapshot model, where the attacker can observe all the database contents at one time instant.
	Note that this excludes timing attacks such as measuring encryption time.
	All security definitions of the schemes and protocols that we discuss here are based on this model.
	Also, the snapshot attacker is the most common attacker that we face today~\cite{secure-queries-overview}.
	The idea is that a hacker or an insider can steal the entire encrypted database and all its contents at some point in time.

	Beyond the snapshot model, it is also possible to consider a stronger adversary who can track communication volume and data access patterns in real time.
	Approaches that help protect against such an attacker include ORAM for protection against access pattern leakage and differential privacy for protection against communication volume leakage.
	Although this model is not a primary target of this paper, our benchmark includes a protocol (Section~\ref{sec:oram}) that is secure in this setting to show the cost of adding such protection.

	We wanted to specifically comment on a work of \textcite{leakage-abuse-grubs-2017}, which demonstrates a series of attacks against OPE and ORE schemes.
	The attacks can be very successful, but they depend on certain prerequisites.
	First, all attacks assume the existence of a well-correlated auxiliary dataset.
	Second, the binomial attack, which works against a ``perfectly secure frequency-hiding scheme'', reliably recovers only high-frequency elements.
	Finally, the attacks are specifically devastating against encrypted strings (e.g.\ first and last names) as opposed to numerical data, and we also do not recommend using OPE / ORE for strings (see Section~\ref{sec:variable-inputs}).
	One of the conclusions of our work is that security is negatively correlated with performance and it is up to a practitioner to trade off security and performance constraints.

	\subsection{A note on variable-length inputs}\label{sec:variable-inputs}

		A generic OPE / ORE scheme accepts bit-strings of any length as inputs, and treats them as numbers or processes them bit-by-bit.
		We warn against supplying raw bytes of variable length (e.g.\ encoded strings) to OPE and ORE schemes, as such an approach will introduce both performance and security challenges.

		From the performance standpoint, the complexity of OPE / ORE schemes usually depends on the input length at least linearly (see Table~\ref{tbl:primitive-usage-theory}).
		32-bit numbers already introduce a noticeable overhead for some (usually more secure) schemes, and supplying arbitrary-length inputs may worsen performance by at least an order of magnitude.

		Security of such a construction will be minimal as most schemes leak some information about the magnitude of the difference, and longer inputs will naturally be treated as larger numbers.
		Thus, the difference between long and short inputs will be apparent.
		We refer to the work of~\textcite{leakage-abuse-grubs-2017} as they have a practically supported discussion of security consequences of using OPE / ORE with arbitrary strings.

		On the other hand, other protocols in our benchmark can usually handle variable-length inputs as long as they fit into a single block for the underlying block cipher.
