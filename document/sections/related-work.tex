\chapter{Related work}\label{section:related-work}
\thispagestyle{myheadings}

	In this section we will go over the relevant work in the area.
 	The discussion is grouped by the query and adversary type, consistent with the overall thesis structure.
	\emph{Some of the following sections were paraphrased or taken verbatim from my published work \cite{ore-benchmark-17,epsolute}.}

	\section{Range query security in a snapshot model}

		In \cref{section:range-snapshot} we explore the methods of answering range queries in a snapshot adversary setting.
		We group these methods into those relying on \acrshort{ope} / \acrshort{ore} schemes coupled with regular database range indices, such as \BPlus{} tree, and other mostly interactive constructions that use custom data structures to support secure indexing.

		The list of recently proposed \acrfull{ope} schemes includes \cite{ope-original, anti-tamper-dbs, bclo-ope, ope-leakage, ope-beyond-one-wayness, ope-early-fh-ope, ope-beyond-ideal-object, ope-mv-opes, fh-ope, ope-mv-popes, ope-multi-user, ope-note, ope-for-encrypted-dbs, ope-structure, ope-non-linear, ope-mdope, ope-outsourced-database, fh-ope-imporved-update}. % chktex 2
		\textcite{ope-ideal-security-protocol} present a nice analysis of these schemes and they are the first to show that using a stateful scheme one can achieve the ideal security guarantees for \acrshort{ope}.
		In addition, there are a number of \acrfull{ore} schemes \cite{ore-original, clww-ore, lewi-wu-ore, parameter-hiding-ore, parameter-hiding-ore, ore-learning, ore-partial, ore-multi-client,  delegatable-ore, multi-client-ore} and related protocols \cite{ore-sorel} that have been proposed. % chktex 2
		After the introduction of \acrshort{ope} / \acrshort{ore} as the means of protecting range indices, a line of attacks \cite{leakage-abuse-attacks-cash-15,attacks-what-else-revealed,inference-attacks-naveed-15, ore-file-injection-attack} have emerged.

		The second group of approaches assumes an outsourced setting where the client may have to communicate with the server during query processing \cite{pope, florian-protocol, secure-queries-overview, practical-range-search}.
		We would like to point out that there are some other methods that can be used to run range queries on encrypted data that use different types of schemes and techniques.
		CryptDB \cite{crypt-db} is a seminal work in this direction offering computations over encrypted data.
		Arx \cite{arx} provides strictly stronger security guarantees by using only semantically secure primitives.
		Seabed \cite{seabed} uses an additively symmetric homomorphic encryption scheme for aggregates and certain filter queries.
		\textcite{ppqed} offer a method to verify and apply a predicate (a junction of conditions) using garbled circuits or homomorphic encryption without revealing the predicate itself.
		SisoSPIR \cite{sisospir} presents a mechanism to build an oblivious index tree such that neither party learns the pass taken.
		See \cite{secure-queries-overview} and \cite{protocols-survey} for an overview of other methods. % chktex 2
		We also note a work on theoretical analysis of \acrshort{ore} security \cite{ore-theory-security}.

	\section{Range query security in a persistent model}

		For the persistent adversary setting, we group the related secure databases, engines, and indices into two categories
		\begin{enumerate*}[label={(\roman*)}]
			\item systems that are oblivious or volume-hiding and do not require \acrfull{tee}, and
			\item constructions that rely on \acrshort{tee} (usually, Intel \acrshort{sgx}).
		\end{enumerate*}
		In this section, where relevant, we aim to compare the related work to our own point and range query solution, \epsolute{} \cite{epsolute}, see \cref{section:range-persistent}.
		\emph{We claim that \epsolute{} is the most secure and practical range- and point-query engine in the outsourced database model, that protects both \acrfull{ap} and \acrfull{cv} using \acrlong{dp}, while not relying on \acrshort{tee}, linear scan or padding result size to the maximum.}

		\subsection{Obliviousness and volume-hiding without enclave}

			This category is the most relevant to \epsolute{}, wherein the systems provide either or both \acrshort{ap} and \acrshort{cv} protection without relying on \acrshort{tee}.
			\crypte{} \cite{crypte} is a recent end-to-end system executing ``\acrshort{dp} programs''.
			\crypte{} has a different model than \epsolute{} in that it assumes two non-colluding servers, an adversarial querying user (the analyst), and it uses \acrshort{dp} to protect the privacy of an individual in the database, which includes volume-hiding for aggregate queries.
			\crypte{} also does not consider oblivious execution and attacks against the \acrshort{ap}.
			Shrinkwrap \cite{shrinkwrap} (and its predecessor SMCQL \cite{smcql}) is an excellent system designed for complex queries over federated and distributed data sources.
			In Shrinkwrap, \acrshort{ap} protection is achieved by using oblivious operators (linear scan and sort) and \acrshort{cv} is concealed by adding fake records to intermediate results with \acrshort{dp}.
			Padding the result to the maximum size first and doing a linear scan over it afterwards to ``shrink'' it using \acrshort{dp}, is much more expensive than in \epsolute{}, however.
			In addition, in processing a query, the worker nodes are performing an $O(n \log{n})$ cost oblivious sorting, where $n$ is the maximum result size (whole table for range query), since they are designed to answer more general complex queries.
			SEAL \cite{seal} offers adjustable \acrshort{ap} and \acrshort{cv} leakages, up to specific bits of leakage.
			SEAL builds on top of Logarithmic-SRC \cite{practical-range-search}, splits storage into multiple \acrshortpl{oram} to adjust \acrshort{ap}, and pads results size to a power of 2 to adjust \acrshort{cv}.
			\epsolute{}, on the other hand, fully hides the \acrshort{ap} and uses \acrshort{dp} with its guarantees to pad the result size.
			PINED-RQ \cite{pined-rq} samples Laplacian noise right in the B+ tree index tree, adding fake and removing real pointers according to the sample.
			Unlike \epsolute{}, PINED-RQ allows false negatives (i.e., result records not included in the answer), and does not protect against \acrshort{ap} leakage.
			On the theoretical side, \textcite{differential-obliviousness} (followed by \textcite{differential-obliviousness-followup}) treat the \acrshort{ap} itself as something to protect with \acrshort{dp}.
			\cite{differential-obliviousness} introduces a notion of differential obliviousness that is admittedly weaker than the full obliviousness used in \epsolute{}. % chktex 2
			Most importantly, \cite{differential-obliviousness} ensures differential privacy with respect to the \acrshort{oram} only, while \epsolute{} ensures \acrshort{dp} with respect to the entire view of the adversary. % chktex 2

		\subsection{Enclave-based solutions}

			Works in this category use trusted execution environment (usually, \acrshort{sgx} enclave).
			These works are primarily concerned with the \acrshort{ap} protection for both trusted and untrusted memory, unlike \epsolute{} which also protects \acrshort{cv}.
			Cipherbase \cite{cipherbase-daas} was a pioneer introducing the idea of using \acrshort{tee} (\acrshort{fpga} at that time) to assist with \acrshort{rdbms} security.
			HardIDX \cite{hardidx} simply puts the B+ tree in the enclave, while StealthDB \cite{stealth-db} symmetrically encrypts all records and brings them in the enclave one at a time for processing.
			EnclaveDB \cite{enclave-db} assumes somewhat unrealistic \SI{192}{\giga\byte} enclave and puts the entire database in it.
			ObliDB \cite{oblidb} and Opaque \cite{opaque} assume fully oblivious enclave memory (not available as of today) and devise algorithms that use this fully trusted portion to obliviously execute common \acrshort{rdbms} operators, like filters and joins.
			Oblix \cite{oblix} provides a multimap that is oblivious both in and out of the enclave.
			HybrIDX claims protection against both \acrshort{ap} and \acrshort{cv} leakages, but unlike \epsolute{} it only obfuscates them.
			\epsolute{} offers an indistinguishability guarantee for \acrshort{ap} and a \acrshort{dp} guarantee for \acrshort{cv}, while HybrIDX hides the exact result size and only obfuscates the \acrshort{ap}.
			Lastly, Hermetic \cite{hermetic} takes on the \acrshort{sgx} side-channel attacks, including \acrshort{ap}.
			It provides oblivious primitives, however, it only offers protection against software and not physical attacks (e.g., it trusts a hypervisor to disable interrupts).

	\section{\texorpdfstring{\acrshort{knn}}{kNN} query security in a snapshot model}\label{section:related-work:knn}

		In this section, where relevant, we aim to compare the related work to our own \acrshort{knn} query solution, \kanon{} \cite{k-anon}, see \cref{section:knn-snapshot}.

		A work immediately related to ours is QuickN by \textcite{quick-n}.
		QuickN offers an adaptation of nearest-neighbor search algorithm in conventional tree data structures (i.e., R-trees) to well-established \acrfull{ope} schemes.
		Unlike our solution that involves a novel property-preserving encryption scheme specifically designed for high-dimensional vectors, QuickN encrypts each vector dimension separately with \acrshort{ope}.
		\cite{quick-n} includes the experiments with attacks against their solution (attack by \textcite{leakage-abuse-grubs-2017}) and report a high degree of protection against them (at most \SI{3.7}{\percent} matching rate for the inference attacks against coordinates represented as pairs of longitude and latitude).

		QuickN approach of applying \acrshort{ope} to an R-tree, however, has some disadvantages.
		First, an ideal stateless \acrshort{ope} has been shown inferior (\cite{ope-leakage}) to its counterpart, an \acrfull{ore} in which the comparison over ciphertexts is defined explicitly.\footnote{
			\cite{quick-n} uses mOPE \cite{ope-ideal-security-protocol} which is an interactive protocol and not a traditional lightweight stateless \acrshort{ope} like \cite{bclo-ope}.
			Since mOPE is an ideal (though stateful) \acrshort{ope}, \textcite{quick-n} do not include \acrshort{ope} leakage in their security definition.
		}
		An \acrshort{ore}, in turn, can have a varying level of security, with the higher security level translating into lower comparison performance \cite{ore-benchmark-17}.
		In QuickN, an R-tree-based nearest-neighbor algorithm involves a very high number of comparisons, linear in data dimensionality.
		With the cost of comparison no longer negligible, the overhead of a query over 2D or 3D is already high, saying nothing of 768-dimensional vectors that our work targets.
		Second, QuickN protocol is not single-round (i.e., it takes two roundtrips per query) and it returns a large number of false positive results even for a minimal $k$ (\num{4000} false positives for $10^6$ dataset and $k = 1$).

		\textcite{knn-aspe} propose a novel scheme, \acrshort{aspe}, that preserves a special type of scalar product.
		Namely, \acrfull{aspe} scheme preserves the scalar product of a database point and a query point, but not the product of a database point with itself or another database point.
		\acrshort{aspe} is then naturally integrated in existing \acrshort{knn} algorithms that use such asymmetric scalar product in their indices.
		The work of \textcite{knn-aspe} is similar to \kanon{} in that we also apply a property-preserving encryption scheme to existing \acrshort{knn} algorithms.
		\acrshort{aspe} is different in that it preserves a scalar product while we preserve an $\text{L}_2$ distance comparison, and \acrshort{aspe} has been broken in \cite{secure-nn-revisited-break-aspe} (although the attack is a chosen plaintext attack, i.e., one cannot decrypt a random ciphertext).

		Other works either target different aspects of query security, like integrity and soundness of results \cite{knn-integrity-soundness,svknn}, or involve mechanisms other than property-preserving encryption \cite{seceqp,practical-approx-knn,knn-sharing-keys,knn-mult-data-owners,knn-over-encrypted,knn-paillier,knn-blind,knn-homomorphism,knn-strong-location-privacy,knn-no-anonymizers,knn-efficient,knn-new-casper}.
