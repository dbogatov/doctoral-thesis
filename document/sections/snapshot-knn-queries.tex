\chapter{\texorpdfstring{\acrshort{knn}}{kNN} queries in the snapshot model}\label{section:knn-snapshot}
\thispagestyle{myheadings}

	In this section I analyze secure \acrshort{knn} queries in the snapshot adversary model.
	I study the effect of protecting the records with a type of property-preserving encryption on quality of search and efficiency of certain attacks.
	Specifically, I examine theoretically and practically how accuracy of both \acrshort{knn} search and \acrshort{ml}-based inversion attack degrade with added security.

	\section{Introduction}

		Nearest-neighbor search is a type of optimization problem that, given a set of objects and a distance metric, requires finding the object closest to a given object according to the distance metric.
		A \acrfull{knn} search is a subtype of a general nearest-neighbor problem where $k$ closest objects are requested.
		Applications that use \acrshort{knn} search only need to define the objects and the metric.
		For example, a street map application would define the 2D coordinates of the buildings as objects and Euclidean distance as a metric, then the query could be ``give 5 restaurants closest to the current user position''.
		A document search application would define the keyword vector for a document as an object and an inner product distance as a metric, then the query could be ``give 3 documents most similar to the given text'' (similar applications may search images, videos and sounds).

		To run a \acrshort{knn} query securely in an outsourced database model, I propose to use an approach similar to \acrshort{ore} for range queries.
		While to run a range query one needs to be able to \emph{compare} the ciphertexts (exactly what \acrshort{ore} does), to run a \acrshort{knn} query one needs to know \emph{a distance comparison} between pairs of ciphertexts.
		That is, one needs to maintain (or reveal) an order of distances of all pairs of encrypted objects.
		For example, if $x$ was further from $y$ than $z$ was, then the encryption of $x$ needs to be further from the encryption of $y$ than the encryption of $z$ is.

		Although such a \acrfull{dcpe} scheme would allow running the query with absolute accuracy, the construction would suffer from the same security issues that the \acrshort{ore} methods did.
		Most of all, while \acrshort{ore}-encrypted dataset reveals the total order, the \acrshort{dcpe}-encrypted values will reveal \emph{the total relative position} of all elements.
		The exact distances will be hidden (subject to the \acrshort{dcpe} scheme's leakage), but the relative position is a substantial leakage that may lead to reconstruction attacks.
		To mitigate the leakage we may use a form of \emph{an approximate} \acrshort{dcpe}, which ideally would add controlled noise or inaccuracy to the relative distances.

	\section{Related Work}

		Work in this area is naturally split into two groups --- mechanisms and constructions that offer certain security guarantees for nearest-neighbor queries, and attacks against those constructions.

		A work immediately related to ours is that of \textcite{quick-n}.
		QuickN offers an adaptation of nearest-neighbor search algorithm in conventional tree data structures (i.e., R-trees) to well-established \acrfull{ope} schemes.
		Unlike our solution that involves a novel property-preserving encryption scheme specifically designed for high-dimensional vectors, QuickN encrypts each vector dimension separately with \acrshort{ope}.
		\cite{quick-n} includes the experiments with attacks against their solution (attack of \textcite{leakage-abuse-grubs-2017}) and report high degree of protection against them.

		QuickN approach of applying \acrshort{ope} to an R-tree, however, has some disadvantages.
		First, an ideal stateless \acrshort{ope} has been shown inferior (\cite{ope-leakage}) to its counterpart, an \acrfull{ore} in which the comparison over ciphertexts is defined explicitly.\footnote{
			\cite{quick-n} uses mOPE \cite{ideal-ope} which is an interactive protocol and not a traditional lightweight stateless \acrshort{ope} like \cite{bclo-ope}.
			Since mOPE is an ideal (though stateful) \acrshort{ope}, \cite{quick-n} do not include \acrshort{ope} leakage in their security definition.
		}
		An \acrshort{ore}, in turn, can have a varying level of security, with the higher security level translating into lower comparison performance.
		In QuickN, an R-tree-based nearest-neighbor algorithm involves a very high number of comparisons, linear in data dimensionality.
		With the cost of comparison no longer negligible, the overhead of query over 2D or 3D is already high, saying nothing of 768-dimensional vectors that our work targets.
		Second, QuickN protocol is not single-round (i.e., it takes two roundtrips per query) and it returns a large number of false positive results even for a minimal $k$ (\num{4000} false positives for $10^6$ dataset and $k = 1$).

		\textcite{knn-aspe} offer a novel scheme, ASPE, that preserves a special type of scalar product.
		ASPE is then naturally integrated in existing \acrshort{knn} algorithms that rely on a scalar product.
		\cite{knn-aspe} is similar to ours in that we also apply a property-preserving encryption scheme to existing \acrshort{knn} algorithms.
		ASPE is different in that it preserves a scalar product while we preserve an L2 distance comparison, and ASPE has been broken in \cite{secure-nn-revisited-break-aspe} (although an attack is a chosen plaintext attack, i.e., one cannot decrypt a random ciphertext).

		Other works either target different aspects of query security, like integrity and soundness of results \cite{knn-integrity-soundness,svknn}, or involve mechanisms other than property-preserving encryption \cite{seceqp,practical-approx-knn,knn-sharing-keys,knn-mult-data-owners,knn-over-encrypted,knn-paillier,knn-blind,knn-homomorphism,knn-strong-location-privacy,knn-no-anonymizers,knn-efficient,knn-new-casper}.

	\section{\texorpdfstring{\acrlong{dcpe}}{Distance Comparison Preserving Encryption}}

		A candidate approximate \acrshort{dcpe} scheme has been proposed by \textcite{dcpe}.
		The scheme provides the following guarantee on its ciphertexts
		\begin{multline*}
			\forall x, y, z \in \mathbb{X} : \algo{Dist}{x, y} < \algo{Dist}{x, z} - \beta \\
			\implies \algo{Dist}{f(x), f(y)} < \algo{Dist}{f(x), f(z)}
		\end{multline*}
		where $\mathbb{X} \subseteq \mathbb{R}^d$ is the set of $d$-dimensional vectors of real numbers, \algo{Dist} is the inner product distance over elements in $\mathbb{X}$, and $\beta$ is the approximation term.
		Parameter $\beta$ partially defines the security of the encrypted set --- the larger $\beta$, the fewer distance comparisons are preserved, the less accurate the search and the reconstruction attacks would be.
		\textcite{dcpe} prove protection against membership inference attacks \cite{memebership-inference-attacks-knn} (whether an individual is in the database or not), and against approximate frequency-finding attacks (how many times the element appears in the set, see \cite{leakage-abuse-grubs-2017} for \acrshort{ore} frequency attacks).
		As for the choice of $\beta$, \cite{dcpe} proves that $\beta \approx \sqrt{\max N}$ would hide about half of the inputs bits, for $\max N$ being the maximum vector length in the dataset.

		The scheme works by amplifying the input vector length by a secret factor, then constructing a $d$-dimensional $\beta$-radius hyperball, and finally setting the vector's tip to a uniformly sampled point on that ball.
		The decryption uses the same secret seed and thus constructs the same amplification factor and the same ball, then makes the inverse arithmetic steps to derive the original vector.
		The security of the scheme thus depends on the maximum amount of amplification, the radius of the hyperball $\beta$ and the size of seed for the samplers.
		As the construction operates on real numbers, an open question remains on how to avoid negative side-effects of floating point numbers bit representation.
		I refer to the original paper \cite{dcpe} for the detailed $\beta$-\acrshort{dcpe} construction.

	\section{\texorpdfstring{\acrshort{knn}}{kNN} search accuracy}

		With the $\beta$-\acrshort{dcpe} as a component, we can model the protocols similar to \acrshort{ore} ones.
		In the setup protocol \protocolSetup{}, \user{} simply encrypts the entire input, one vector at a time, and sends the encrypted data over to \server{}.
		In the query protocol \protocolQuery{}, \user{} encrypts the query with \acrshort{dcpe}, sends the ciphertext to \server{}, while \server{} runs a standard \acrshort{knn} search against the ciphertext.
		$k$ encrypted vectors are then returned to \user{}, who decrypts them as the last step.
		These protocols run for a single set of secrets including $\beta$.
		To measure the effect of different levels of security on search accuracy, I propose to repeat the experiments for different values of $\beta$.

		For the choice of the dataset, I suggest using the established information retrieval \acrshort{trec} test collections.
		Such collection consists of a set of documents, a set of topics (questions) and a corresponding set of relevance judgments (correct answers).
		To use the test collection, we need to convert the documents and queries into vectors.\footnote{
			Hamed Zamani has collaborated with us and provided the vectorized data and query sets.
		}
		A benefit of using a \acrshort{trec} dataset is being able to evaluate relevant metrics over the produced results, for example, \acrshort{mrr} \cite{mrr} and \acrshort{dcg} \cite{dcg}.
		We can then track how these metrics, along with the simpler edit distance and set difference over the result, degrade with higher security.

		Lastly, for an actual implementation I suggest using an existing component for the bare \acrshort{knn} search.
		\acrshort{faiss} \cite{faiss} is a \acrshort{gpu}-enabled library for efficient similarity search and clustering of dense vectors.
		Given that the \acrshort{trec} vectors are $d = 768$ dimensional with a maximum Euclidean length of 11, \acrshort{faiss} seems to be an ideal candidate.

	\section{Security against attacks}
