
% cSpell:ignore Camenisch Drijvers Dubovitskaya

\chapter{Abstract of \cite{bogatov-idemix-2020}}\label{section:appendix:idemix-abstract}
\thispagestyle{myheadings}

	In permissioned blockchain systems, participants are admitted to the network by receiving a credential from a certification authority.
	Each transaction processed by the network is required to be authorized by a valid participant who authenticates via her credential.
	Use case settings where privacy is a concern thus require proper privacy-preserving authentication and authorization mechanisms.

	Anonymous credential schemes allow a user to authenticate while showing only those attributes necessary in a given setting.
	This makes them a great tool for authorizing transactions in permissioned blockchain systems based on the user's attributes.
	In most setups, there is one distinct certification authority for each organization in the network.
	Consequently, the use of plain anonymous credential schemes still leaks the association of a user to the organization that issued her credentials.
	Camenisch, Drijvers and Dubovitskaya \cite{delegatable-creds} therefore suggest the use of a delegatable anonymous credential scheme to also hide that remaining piece of information.

	In this paper, we propose the revocation and auditability --- two functionalities that are necessary for real-world adoption --- and integrate them into the scheme.
	We present a complete protocol, its security definition and the proof, and provide its open-source implementation.
	Our distributed-setting performance measurements show that the integration of the scheme with Hyperledger Fabric, while incurring an overhead in comparison to the less privacy-preserving solutions, is practical for settings with stringent privacy requirements.
