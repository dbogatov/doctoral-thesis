
% cSpell:ignore

\chapter{Abstract of \cite{dispot}}\label{section:appendix:dispot-abstract}
\thispagestyle{myheadings}

	\noindent \textbf{Motivation:}
		The complexity of protein-protein interactions (PPIs) is further compounded by the fact that an average protein consists of two or more domains, structurally and evolutionary independent subunits.
		Experimental studies have demonstrated that an interaction between a pair of proteins is not carried out by all domains constituting each protein, but rather by a select subset.
		However, finding which domains from each protein mediate the corresponding PPI is a challenging task.

	\noindent \textbf{Results:}
		Here, we present Domain Interaction Statistical POTential (DISPOT), a simple knowledge-based statistical potential that estimates the propensity of an interaction between a pair of protein domains, given their SCOP family annotations.
		The statistical potential is derived based on the analysis of more than \num{352000} structurally resolved protein-protein interactions obtained from DOMMINO, a comprehensive database on structurally resolved macromolecular interactions.

	\noindent \textbf{Availability and implementation:}
		DISPOT is implemented in Python 2.7 and packaged as an open-source tool.
		DISPOT is implemented in two modes, \emph{basic} and \emph{auto-extraction}.
		The source code for both modes is available on \href{https://github.com/korkinlab/dispot}{GitHub} and standalone docker images on \href{https://hub.docker.com/r/korkinlab/dispot}{DockerHub}.
		The web-server is freely available at \href{http://dispot.korkinlab.org/}{dispot.korkinlab.org}.

	\noindent \textbf{Contact:}
		\href{korkin@korkinlab.org}{korkin@korkinlab.org} or \href{onarykov@wpi.edu}{onarykov@wpi.edu}

	\noindent \textbf{Supplementary information:}
		\href{https://academic.oup.com/bioinformatics/article-lookup/doi/10.1093/bioinformatics/btz587#supplementary-data}{Supplementary data} are available at \textit{Bioinformatics} online.
