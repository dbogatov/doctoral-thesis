% cSpell:disable

\usepackage[a4paper, margin=1in]{geometry}

\usepackage{amsmath}		% allow \text{} in math mode
\usepackage{amsfonts}
\usepackage{bm}
\usepackage{listings}
\usepackage{subcaption}
\usepackage{graphicx}
\usepackage{balance}
\usepackage{hyperref}
\usepackage{mathtools}
\usepackage{breqn}
\usepackage{nicefrac}
\usepackage{booktabs}
\usepackage{multirow}

\usepackage{color}
\usepackage{marginnote}
\setlength{\marginparwidth}{1.5cm}
\renewcommand*{\marginfont}{\color{red}\tiny}

\let\proof\relax
\let\endproof\relax
\usepackage{amsthm}
\let\proof\relax
\let\endproof\relax

\usepackage{caption} 

\usepackage[all]{nowidow}
\usepackage{hyphenat}
\usepackage{enumitem}

\usepackage[
	backend=biber,
	style=numeric,
	giveninits=true,
	sorting=nyvt,
	maxbibnames=1000
]{biblatex}

\DeclareFieldFormat%
	[inproceedings]
	{booktitle}{\textit{#1}}

\DeclareFieldFormat%
	[article]
	{journal}{\textit{#1}}

\bibliography{bibfile}

\graphicspath{{./graphics/}}

% https://tex.stackexchange.com/a/226857/97712
\makeatletter
\let\oldmarginnote\marginnote%
\renewcommand*{\marginnote}[1]{%
	\begingroup%
	\ifodd\value{page}
		\if@firstcolumn\reversemarginpar\fi
	\else
		\if@firstcolumn\else\reversemarginpar\fi
	\fi
	\oldmarginnote{#1}%
	\endgroup%
}
\makeatother


\newcommand{\BigO}[1]{\mathcal{O}\left(#1\right)}
\newcommand{\BigOmega}[1]{\Omega\left(#1\right)}

% custom theorems if needed
\newtheoremstyle{mytheor}
	{1ex}{1ex}{\normalfont}{0pt}{\scshape}{.}{1ex}
	{{\thmname{#1 }}{\thmnumber{#2}}{\thmnote{ (#3)}}}

\theoremstyle{mytheor}

\newtheorem{thm}{Theorem}
\newtheorem{lem}[thm]{Lemma}
\newtheorem{cor}[thm]{Corollary}
\newtheorem{rem}[thm]{Remark}
\newtheorem{remark}[thm]{Remark}
\newtheorem{conj}[thm]{Conjecture}
\newtheorem{definition}[thm]{Definition}
\newtheorem{claim}[thm]{Claim}
\newtheorem{note}[thm]{Note}
\newtheorem{assume}[thm]{Assumption}

\DeclarePairedDelimiter\ceil{\lceil}{\rceil}
\DeclarePairedDelimiter\floor{\lfloor}{\rfloor}

\lstset{ 
	mathescape=true,
	tabsize=2,
	morekeywords={return,then,if,else,foreach,endforeach}
}

% \captionsetup[table]{skip=10pt}
% \captionsetup[figure]{skip=10pt}

% \setlength{\parindent}{0pt}
% \setlength{\parskip}{3pt}

% \newenvironment{myitemize}
% {	\begin{itemize}
% 		\setlength{\itemsep}{0pt}
% 		\setlength{\parskip}{0pt}
% 		\setlength{\parsep}{0pt}	}
% {	\end{itemize}					}

% \newenvironment{myitemizenoindent}
% {	\begin{itemize}[leftmargin=*]
% 		\setlength{\itemsep}{0pt}
% 		\setlength{\parskip}{0pt}
% 		\setlength{\parsep}{0pt}	}
% {	\end{itemize}					}

% \newenvironment{myenumerate}
% {	\begin{enumerate}
% 		\setlength{\itemsep}{0pt}
% 		\setlength{\parskip}{0pt}
% 		\setlength{\parsep}{0pt}	}
% {	\end{enumerate}					} 

% \newenvironment{mydescription}
% {	\begin{description}
% 		\setlength{\itemsep}{0pt}
% 		\setlength{\parskip}{0pt}
% 		\setlength{\parsep}{0pt}	}
% {	\end{description}				} 
