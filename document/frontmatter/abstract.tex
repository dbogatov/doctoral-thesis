As organizations struggle with processing vast amounts of information, outsourcing sensitive data to third parties becomes a necessity.
Various cryptographic techniques are used in outsourced database systems to ensure data privacy while allowing for efficient querying.
This thesis proposes a definition and components of a new secure and efficient outsourced database system, which answers various types of queries, with different privacy guarantees in different security models.

This work starts with the survey of five order-preserving and order-revealing encryption schemes that can be used directly in many database indices, such as the B+ tree, and five range query protocols with various tradeoffs in terms of security and efficiency.
The survey systematizes the state-of-the-art range query solutions in a snapshot adversary setting and offers some non-obvious observations regarding the efficiency of the constructions.

The thesis then proceeds with \epsolute{} --- an efficient range query engine in a persistent adversary model.
In \epsolute{}, security is achieved in a setting with a much stronger adversary where she can continuously observe everything on the server, and leaking even the result size can enable a reconstruction attack.
\epsolute{} proposes a definition, construction, analysis, and experimental evaluation of a system that provably hides both access pattern and communication volume while remaining efficient.

The dissertation concludes with \kanon{} --- a secure similarity search engine in a snapshot adversary model.
The work presents a construction in which the security of \acrshort{knn} queries is achieved similarly to \acrshort{ope} / \acrshort{ore} solutions --- encrypting the input with an approximate \acrlong{dcpe} scheme so that the inputs, the points in a hyperspace, are perturbed, but the query algorithm still produces accurate results.
Analyzing the solution, we run a series of experiments to observe the tradeoff between search accuracy and attack effectiveness.
We use \acrshort{trec} datasets and queries for the search, and track the rank quality metrics such as \acrshort{mrr} and \acrshort{ndcg}.
For the attacks, we build an \acrshort{lstm} model that trains on the correlation between a sentence and its embedding and then predicts words from the embedding.
We conclude on viability and practicality of the solution.
