As organizations struggle with processing vast amounts of information, outsourcing sensitive data to third parties becomes necessary.
Various cryptographic techniques are used in outsourced database systems to ensure data privacy while allowing for efficient querying.
In this prospectus, I propose a definition and components of a new secure and efficient outsourced database system, which answers various types of queries, with different privacy guarantees in different security models.

I start with my survey work on five order-preserving and order-revealing encryption schemes that can be used directly in many database indices, such as the B+ tree, and five range query protocols with various tradeoffs in terms of security and efficiency.
This work systematizes the state-of-the-art range query solutions in a snapshot adversary setting and offers some non-obvious observations regarding the efficiency of the constructions.

I then follow with a recently published work, \epsolute{} --- an efficient range query engine in a persistent adversary model.
In this work, we achieve security in a setting with a much stronger adversary where she can continuously observe everything on the server, and leaking even the result size can enable a reconstruction attack.
We propose a definition, construction, analysis, and experimental evaluation of a system that provably hides both access pattern and communication volume while remaining efficient.

Finally, I analyze the secure \acrlong{knn} queries, in which the security is achieved similarly to \acrshort{ope}/\acrshort{ore} solutions --- encrypting the input with an approximate \acrlong{dcpe} scheme so that the inputs are perturbed, but the query algorithm still produces accurate results.
In this work, we adapt a property-preserving encryption scheme and run a series of experiments to observe the tradeoff between search accuracy and attack effectiveness.
We use \acrshort{trec} datasets and queries for the search and track the accuracy metrics such as \acrshort{mrr} and \acrshort{dcg}.
For the attacks, we build an \acrshort{lstm} model that trains on the correlation between a sentence and its embedding and then predicts words from the embedding.
